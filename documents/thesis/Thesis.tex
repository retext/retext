\documentclass[11pt,a4paper]{article}
\usepackage[paper=a4paper,width=15cm,height=22cm,twoside]{geometry}

\usepackage{xltxtra}

\usepackage{setspace}
\linespread{1.15}
\setlength{\parskip}{0.75em}
\setlength{\parindent}{0em}

\usepackage{fancyhdr}
\pagestyle{fancy}
\fancyhead{}
\fancyfoot{}
\fancyfoot[RO,LE]{\thepage}
\renewcommand{\headrulewidth}{0pt}
\renewcommand{\footrulewidth}{0pt}

\usepackage[german]{babel}
\usepackage{graphicx}
\usepackage{acronym}

\usepackage{hyperref}
\usepackage[usenames,dvipsnames]{xcolor}
\hypersetup{
    bookmarks=true,
    colorlinks=true,
    linkcolor=NavyBlue,
    citecolor=NavyBlue,
    urlcolor=NavyBlue
}

\bibliographystyle{plain} 
\bibdata{parts/bibliothek} 

\usepackage{sectsty}
\allsectionsfont{\sffamily} 

\begin{document}

\setmainfont[Mapping=tex-text]{ITC Stone Sans Std}

\pagenumbering{roman}

\newacro{IKTM}[IKT-Medien]{Informations- und Kommunikationsmedien}

\newcommand{\TODO}{\textcolor{red}{\textbf{TODO}}}
\newcommand{\typoquotes}[1]{„#1“}

% Legt Trennungen für Worte fest, die nicht im Wörterburch enthalten sind

\hyphenation{Kommunikations-medien}

\author{Markus Tacker}
\title{Konzeption und Entwurf eines workflowgesteuerten Systems zur Verwaltung von Textbausteinen für \acl{IKTM}}

\begin{center}

\begin{small}

\textbf{Hochschule RheinMain\\Fachbereich Design Informatik Medien\\Studiengang Medieninformatik}

\vspace{1cm}

\textbf{Bachelor-Thesis\\zur Erlangung des akademischen Grades\\Bachelor of Science – B.Sc.}

\end{small}

\vspace{2cm}

\begin{huge}

\textbf{Konzeption und Entwurf eines workflowgesteuerten Systems zur Verwaltung von Textbausteinen für \acl{IKTM}}

\end{huge}

\end{center}

\setmainfont[Mapping=tex-text,BoldFont={Vollkorn-Bold},ItalicFont={Vollkorn-Italic},BoldItalicFont={Vollkorn-Bold Italic}]{Vollkorn}
\setsansfont[Mapping=tex-text]{ITC Stone Sans Std}

\vspace{8cm}

\begin{tabular}{@{}l l}
vorgelegt von & Markus Tacker\\
am & \today\\
& \\
Referent & Prof. Dr. Jörg Berdux\\
Korreferent & Prof. Thomas Steffen
\end{tabular}

\pagebreak

\section*{Erklärung gem. ABPO, Ziff. 6.4.3}

Ich versichere, dass ich die Bachelor-Thesis selbständig verfasst und keine anderen als
die angegebenen Hilfsmittel benutzt habe.

\vspace{1cm}

\begin{tabular*}{\textwidth}{@{\extracolsep{\fill}}l r@{}}
Offenbach am Main, \today & Markus Tacker
\end{tabular*}

\section*{Verbreitung}

Hiermit erkläre ich mein Einverständnis mit den im folgenden aufgeführten
Verbreitungsformen dieser Bachelor-Thesis:

\begin{tabular*}{\textwidth}{@{\extracolsep{\fill}}l r@{}}
Einstellung der Arbeit in die Hochschulbibliothek mit Datenträger: & nein \\
Einstellung der Arbeit in die Hochschulbibliothek ohne Datenträger: & nein \\
Veröffentlichung des Titels der Arbeit im Internet: & ja \\
Veröffentlichung der Arbeit im Internet: & nein
\end{tabular*}

\vspace{1cm}

\begin{tabular*}{\textwidth}{@{\extracolsep{\fill}}l r@{}}
Offenbach am Main, \today & Markus Tacker
\end{tabular*}

\section*{Satz}

Gesetzt mit \LaTeX{} / XeTeX \\
Schrift: Vollkorn\footnote{\url{http://friedrichalthausen.de/?page_id=411}} von Friedrich Althausen

\section*{Diagramme}

Erstellt mit Graphviz\footnote{\url{http://www.graphviz.org/}} und Google Docs\footnote{\url{https://docs.google.com/}}

\vspace{1cm}

\pagebreak

\section*{Danksagung}

\TODO

Ich danke 

\begin{tabular}{@{}l l}
Carsten Fischer & UX Designer \& Informationsarchitekt bei triplesense GmbH\\
Jorinde Gessner & Information Manager bei Ogilvy \& Mather Deutschland GmbH\\
Jan Lochner & freier Multimedia-Producer\\
Sebastian Nell & Director of USE // Connected Products, Scholz \& Volkmer GmbH\\
Marc Stenzel & Berater / Senior Projektmanager New Media bei media deluxe e.K.
\end{tabular}

die mir als Interviewpartner zur Verfügung standen.

\pagebreak

\section*{Zusammenfassung}

Nahezu alle \ac{IKTM} haben eines gemeinsam: sie beinhalten Text. Obwohl viele Personen bei der Erstellung dieser Texte beteiligt sind, werden sie in der Regel in Office-Dokumenten verwaltet, meistens mit Word, bei großen Projekten kommt Excel zu Einsatz. Der Workflow von einem Bearbeiter zum nächsten erfolgt über den Austausch des Office-Dokumentes via E-Mail, Ticketsysteme oder sonstige System zum asynchronen Dateiaustausch wie z.B. Dropbox. Dieser Prozess ist aufwendig und fehleranfällig. Sobald mehrere Personen gleichzeitig an den Texten arbeiten, wird manuelles Eingreifen notwendig um die gemachten Änderungen zusammenzuführen. Aufgrund der Vielzahl der am Text beteiligten Personen sind Office-Dateien ein denkbar schlecht geeignetes Mittel um Texte und ihre Änderungen sauber und nachvollziehbar zu verwalten. Auch das Übertragen von Texten aus Office-Dokumenten ist eine Fehlerquelle – es ist stupides Copy\&\-Paste. In den meisten Fällen müssen dabei im Dokument vorgenommene Formatierungen wie Umbrüche und Absätze entfernt werden um eine saubere Darstellung im Endprodukt zu gewährleisten. Gerade der Text ist der Bestandteil eines \ac{IKTM}, der oft bis zur letzten Minute geändert wird – egal wie viel Aufwand vorher in die Planung geflossen sind. Dies liegt unter anderem daran, dass Text im Gegensatz zu Grafiken, Fotos und anderen Multimedia-Elementen als einziger Informationsträger eindeutig ist und üblicherweise keinen Interpretationsspielraum offen lassen soll. So bietet er auch aus rechtlicher Sicht den problematischsten Bestandteil des Produkts.

Diese Bachelor-Thesis analysiert das beschriebene Problem und konzipiert einen passgenauen Worflow, der alle Beteiligten entsprechend ihrer Aufgabe und Anforderungen integriert. Als Proof-of-Concept wird eine webbasiertes Anwendung entworfen, die den konzipierten Workflow soweit abbildet, dass das Konzept am Beispiel einer Informationsbroschüre überprüft werden kann. 




\pagebreak

\tableofcontents

\pagebreak

\setcounter{page}{1}
\pagenumbering{arabic}

\section{Problem-Analyse}\label{l:problemanalyse}

\begin{figure}[htb]
\begin{center}
\includegraphics[width=\textwidth]{media/chart-2.pdf}
\end{center}
\caption{Bei der Erstellung von Texten beteiligte Personen}
\label{chart:2}
\end{figure}

In diesem Kapitel werden die Probleme beschrieben, die bei der Erstellung von In"-for"-ma"-ti"-ons- und Kom"-mu"-ni"-ka"-ti"-ons"-me"-di"-en in Zusammenhang mit den dargestellten Texten auftreten. Die Abschnitte \ref{l:besondererolle} und \ref{l:werkzeugwahl} analysieren die besondere Rolle von Text und die verwendeten Werkzeuge zu dessen Verwaltung, anschließend zeigt Abschnitt \ref{l:officeprobleme} · S.\pageref{l:officeprobleme} typische Probleme auf, die im Verlauf von Projekten entstehen. Abschnitt \ref{l:praxisbeispiele} · S.\pageref{l:praxisbeispiele} belegt dies mit Beispielen aus der Praxis. 

\bigskip

Die Analyse des Problems basiert auf im April 2012 geführten Interviews mit Personen, die in ihrem Arbeitsalltag regelmäßig mit Texten zu tun haben. Eine Liste der interviewten Personen findet sich in Tabelle \ref{table:interviewpartner} · S.\pageref{table:interviewpartner}.

\subsection{Die besondere Rolle von Text in In"-for"-ma"-ti"-ons- und Kom"-mu"-ni"-ka"-ti"-ons"-me"-di"-en}\label{l:besondererolle}

Es existieren nahezu keine Medien, die ohne Texte auskommen, denn Text ist im Gegensatz zu Grafiken, Fotos oder Animationen ein eindeutiger Informationsträger und unterliegt viel weniger stark einer Interpretation durch den Rezipienten eines Mediums als die symbolisierte oder stilisierte Darstellung von Informationen in audiovisuellen Medien. Text wird in der Marketing-Kommunikation als Unterstützung der zu übermittelnden Information verwendet. Hat man die Aufmerksamkeit des Betrachter eines Produkts erlangt, liefert Text weitere Informationen zum Produkt, er dient dazu, die emotionale Botschaft zu erläutern und zu präzisieren. Auch aus rechtlichen Aspekten ist Text aus den genannten Gründen der einzige verbindliche Informationsträger -- bestes Beispiel hierfür ist das sogenannte \typoquotes{Kleingedruckte}, dass sich gerade bei inhaltlich sehr stark komprimierten Werbeformen, wie z.B. Plakat- oder Fernsehwerbung, findet. Ist die Textmenge, die in der Marketing-Kommunikation zum Einsatz kommt, noch überschaubar, existieren Medien die hauptsächlich aus Text bestehen. Hierunter fallen klassische Druckerzeugnisse wie Broschüren und Kataloge oder Produkte der Unternehmenskommunikation wie Jahresberichte und Pressemeldungen. Besonders digitale Medien werden oft mit großen Textmengen versehen -- von der einfachen Produkt-Microsite\footnote{Internetseite mit sehr kompaktem Inhaltsumfang für einen speziellen Zweck (z.B. Produkt, Gewinnspiel) oder einen besonderen Aktionszeitraum}, über Werbemittel wie Newsletter bis zur Unternehmenswebsite -- die Möglichkeit Inhalte hierarchisch zu strukturieren und sogar über eine Suche zugänglich zu machen hebt eine medienbedingte Limitierung des Umfangs, wie bei Druckprodukten, praktisch auf.

Betrachtet man die Abläufe von Projekten, in deren Verlauf Medien erstellt werden, lassen sich bezüglich der Textbestandteile dieser Produkte immer wieder sehr ähnliche Vorgehensweisen und Besonderheiten beobachten. Aufgrund der verbindlichen Natur von Text sind an der Erstellung der Texte für das Medium mehr Personen beteiligt, als es z.B. für die Gestaltung, die Auswahl von Bildmaterial oder für die Programmierung der Fall ist, da Text sehr viele verschieden Kriterien erfüllen muss. Tabelle \ref{table:textkriterien} · S.\pageref{table:textkriterien} listet exemplarisch eine typische Gruppe von Personen auf, die im Verlauf des Projekts Einfluss auf den Text eines Produkts haben. Dieser Einfluss wird dabei in der Regel nicht in optimaler Reihenfolge und im Sinne des geplanten Projektverlaufes ausgeübt. Gerade auf die Mitarbeiter auf Kundenseite haben Agenturen keinen Einfluss; in Projektplänen lassen sich zwar verbindliche Termine für die Lieferung von Texten des Kunden festlegen, dies verhindert aber keinesfalls, dass zu einem späteren Zeitpunkt Änderungen notwendig werden -- Hinweise von Anwälten sollten im besten Fall \emph{vor} einer Übersetzung vorliegen, diese richten sich aber vorrangig nach ihren eigenen Terminplänen. Auch die Kriterien wie Text beeinflusst wird, sind sehr vielfältig: Im Entwurf und in der Umsetzung der Produkte legen Designer, Informations-Architekten und Produzenten die Struktur von Text wie Art der Ansprache, maximale Wortlänge, Anzahl der Wörter einer Überschrift fest oder diese werden durch das verwendete Medium vorgegeben, Texter legen die Inhalte fest, die wiederum durch Wünsche des Kunden beeinflusst werden; das Lektorat, Fachabteilungen und Anwälte begutachten die Texte dann bezüglich der jeweils erforderlichen Korrektheit.

\begin{table}
\begin{center}
\begin{tabular}{@{}l l l l}
\textbf{Kriterium} & \textbf{Art} & \textbf{Verantwortlich}\\
\hline
Aufgabenverteilung & Mitarbeiter & Projektleiter, Agentur \\
\hline
Zielgruppe & Struktur & Informationsarchitektur\\
\hline
Umfang, Satzlänge & Struktur & Art-Direktion\\
\hline
Länge einzelner Wörter & Struktur & Programmierer\\
\hline
Information & Inhalt & Texter\\
\hline
Orthographie & Korrektheit & Lektorat\\
\hline
Übersetzung & Sprache & Übersetzungsbüro\\
\hline
Suchmaschinen-Optimierung & Inhalt & SEO-Experte\\
\hline
Aufgabenverteilung & Mitarbeiter & Projektleiter, Kunde\\
\hline
Fachliche Aspekte & Korrektheit & Fachabteilung\\
\hline
Rechtliche Aspekte & Korrektheit & Rechtsanwalt\\
\hline
Werbeaussagen & Inhalt & Marketingabteilung\\
\hline
… & … & …
\end{tabular}
\caption{Kriteren von Textbausteinen und verantwortliche Personen}
\label{table:textkriterien}
\end{center}
\end{table}

Wie man Tabelle \ref{table:textkriterien} entnehmen kann, existieren vielfältige Einflussmöglichkeiten auf die Gestaltung von Texten für Medien, die sich auf viele Verantwortliche verteilen. Der Grund dafür ist, dass alle Beteiligten jeweils spezifisches Fachwissen in den Text einfließen lassen, seien es gestalterische Aspekte, die Einfluss auf die Struktur haben, oder das Wissen über exakte technische Abläufe, die nur Spezialisten in den Fachabteilungen auf Kundenseite bekannt sind. Dieses Expertenwissen kann nicht für die meist kurze Projektlaufzeit an die umsetzenden Agentur vermittelt werden. Es ist also unvermeidlich, dass Text während des gesamten Projektverlaufs geändert werden kann. Neben den Einflüssen durch Experten gibt es auch projektbedingte Einflüsse auf Text in letzter Minute. Sind in Texten Informationen enthalten, die einen zeitlichen Aspekt abbilden, ergeben sich durch Verzögerungen im Projekt automatisch Änderungsanforderungen. Ein Beispiel sind Gewinnspiele: Verschiebt sich durch Probleme während des Projekts der Zeitpunkt, ab dem ein Produkt beim Rezipienten vorliegt, müssen auch eventuell knapp kalkulierte Gewinnspieltermine angepasst werden. Ein weiterer Grund für vielfältige Textänderungen im Verlauf eines Projekts ist die Erwartungshaltung des Kunden -- da es Kunden aus ihrem eigenen Arbeitsalltag gewöhnt sind mit Textverarbeitungsprogramme zu arbeiten, und sie so aus eigener Erfahrung vermeintlich wissen dass Texte schnell geändert sind, erwarten sie auch, dass die Texte im Produkt bis zum Schluss geändert werden können; ihnen ist nicht bewusst, das vom ursprünglichen Text im Quelldokument bis zur Darstellung im fertigen Produkt viele aufwändige Arbeitsschritte nötig sein können.

\subsection{Das Werkzeug der Wahl zur Verwaltung von Text: \trademark{Word} und \trademark{Excel}}\label{l:werkzeugwahl}

Zur Abbildung der komplexen Abläufe bei der Erstellung von In"-for"-ma"-ti"-ons- und Kom"-mu"-ni"-ka"-ti"-ons"-me"-di"-en liefern etablierte Software-Hersteller passende Lösungen auch speziell für Texte: Mit \trademark{InCopy} liefert \trademark{Adobe} eine \citequotes{Lösung für Texterstellung und -bearbeitung, die aufgrund der engen Integration mit Adobe InDesign® CS5.5 effektivere Zusammenarbeit zwischen Redakteuren und Layoutern ermöglicht} \cite{adobeincopy} und  die \trademark{Content Station} von \trademark{Woodwing} \citequotes{ist […] eine einzige Oberfläche für alle Schritte des Publishing-Prozesses. […] Unter Nutzung der Desktop- oder der Web-Version können die Team-Mitglieder unabhängig ihres Aufenthaltsorts mitarbeiten} \cite{woodwing} -- um nur zwei Beispiele zu nennen. Doch obwohl spezialisierte Werkzeuge existieren, findet man diese in Agenturen nur selten -- das Werkzeug der Wahl zur Verwaltung der Texte ist in der Regel eine in der Agentur vorhandene Textverarbeitungs- oder Tabellenkalkulationssoftware, in den allermeisten Fällen handelt es sich dabei um den Marktführer in diesem Bereich: \trademark{Microsoft} \trademark{Word} oder \trademark{Excel}. Auf die Probleme, die durch deren Einsatz entstehen, wird im nachfolgenden Abschnitt \ref{l:officeprobleme} · S.\pageref{l:officeprobleme} eingegangen. Zuerst muss jedoch untersucht werden, warum statt spezieller Werkzeuge, die für den komplizierten Workflow in Projekten entwickelt wurden, \trademark{Word} oder \trademark{Excel} eingesetzt werden.

\bigskip

Oberflächlich betrachtet, bieten Textverarbeitungsprogramme die notwendigen Funktionen, um Texte zu verwalten und sind damit scheinbar die natürliche Wahl. Die verwendeten Funktionen sind dabei nachfolgend beschrieben.

\begin{figure}[htb]
\begin{center}
\includegraphics[width=\textwidth]{media/Textbooklet-Word-Dokument.pdf}
\end{center}
\caption{\trademark{Word}-Dokument mit Texten für eine Internetseite}
\label{f:wordbooklet}
\end{figure}

\begin{figure}[htb]
\begin{center}
\includegraphics[width=\textwidth]{media/Textbooklet-Excel-Dokument.pdf}
\end{center}
\caption{\trademark{Excel}-Dokument mit Texten für eine Internetseite}
\label{f:excelbooklet}
\end{figure}

\paragraph{Strukturierung von Texten} Die Möglichkeit, Texte hierarchisch in Dokumente, Seiten, Kapitel, Abschnitte oder Absätze zu unterteilen ermöglicht es die Textbausteine für ein Produkt geordnet zu Erfassen. Neben den eigentlichen Texten lassen sich auch Zusatzinformationen wie die Klasse des Textes dort zuzuordnen. Abbildung~\ref{f:wordbooklet} · S.\pageref{f:wordbooklet} zeigt beispielhaft ein \trademark{Word}-Dokument, in dem die Texte für eine Website definiert werden. Im Dokument existiert pro Seite der Internetpräsenz jeweils ein Abschnitt, der alle Texte auf der Seite beschreibt. Dort finden sich die Texte zu den Platzhaltern, die im Quellcode der Website verwendet werden, die dann an deren Stelle durch entsprechende Methoden einer Template-Engine eingefügt werden. Über die Formatierung der Überschriften im Dokument wird die Hierarchie der Texte definiert. 

Die Verwendung von Tabellen statt Dokumenten ist eine weitere Möglichkeit die verwendeten Texte zu erfassen. Abbildung~\ref{f:excelbooklet} · S.\pageref{f:excelbooklet} zeigt beispielhaft ein \trademark{Excel}-Dokument, in dem pro Zeile ein Text definiert wird. In den Spalten finden sich neben dem eigentlichen Text Zusatzinformationen wie z.B. die Text-Klasse. Tabellarische Dokumente werden oft bei umfangreichen Projekten oder dann verwendet, wenn mehrere Sprachversionen verwaltet werden müssen.

\paragraph{Rechtschreibkorrektur} In Textverarbeitungsprogrammen sind ausgefeilte Funktionen zur Rechtschreibkorrektur enthalten, die bereits während der Eingabe auf Fehler aufmerksam machen und für viele Sprachen verfügbar sind. So ist sichergestellt, dass bereits die erste Version eines Textes relativ wenige Fehler enthält.

\paragraph{Kommentare} Es ist möglich in \trademark{Word}- und \trademark{Excel}-Dokumenten Kommentare zu hinterlassen. Diese werden gesondert hervorgehoben und können zum Austausch über den Text oder für Hinweise zu dessen Verwendung hinterlegt werden und von allen Bearbeitern eingesehen werden.

\paragraph{Änderungsverfolgung} Wenn die Änderungsverfolgung aktiviert ist, werden alle Änderungen an einem Dokument aufgezeichnet. Diese Information kann dabei helfen mehrere Versionen eines Dokumentes manuell zusammenzuführen oder Änderungen an Inhalten vorzuschlagen, zu prüfen und selektiv zu übernehmen.

\paragraph{Verzeichnisse} In \trademark{Word}-Dokumenten ist es möglich, Verzeichnisse wie z.B. ein Inhaltsverzeichnis anzulegen. Dies hilft bei größeren Projekten einen Überblick über den Aufbau des Produkts zu erhalten, sofern die Inhalte mit den passenden Formatvorlagen versehen wurden.

\paragraph{Suchen \& Ersetzen} Da sich die Texte in einem großen Dokument befinden können mit den Funktionen zum Suchen \& Ersetzen schnell bestimmte Inhalte gefunden und angepasst werden.

\paragraph{Export} Die \trademark{Office}-Programme verfügen über die Möglichkeit des Exports in verschiedene Formate. Verwendet wird vor allem PDF bei \trademark{Word}-Dokumenten zur Abstimmung, unter anderem auch deswegen, weil es in \trademark{Adobe} \trademark{Acrobat} umfangreiche Korrekturfunktionen gibt. Bei \trademark{Excel}-Dokumenten wird vor allem CSV verwendet, da Texte in diesem Format leicht in andere Systeme importiert werden können.

\paragraph{Formatierungsfunktionen} Umfangreiche Formatierungsfunktion erlauben es Texte besonders auszuzeichnen. Oft werden durch farbige Markierung Passagen markiert die entfallen oder inhaltlich überarbeitet werden müssen. Auch können Formatierungen so angelegt sein wie sie im Produkt erscheinen sollen -- in der Regel werden dann Teile des Textes oder einzelne Wörter fett oder kursiv formatiert. Funktionen zum Setzen von Hyperlinks werden gerade bei Web-Projekten verwendet um Links zu definieren die im Produkt verwendet werden sollen.

\bigskip

Auch im Hinblick auf nicht-funktionale Aspekte bieten Textverarbeitungsprogramme einige Vorteile, sind sie doch in den allermeisten Unternehmen der Standard zur Textverarbeitung und sogar plattformunabhängig verfügbar -- zumindest existiert die Möglichkeit das \trademark{Microsoft Office}-Dateiformat auf allen Plattformen zu bearbeiten. Da praktisch bei allen Projektbeteiligten eine Installation von \trademark{Microsoft Office} vorausgesetzt werden kann, werden \trademark{Word} und \trademark{Excel} zu \typoquotes{leichtgewichtigen} Werkzeugen, die vom Anwender keine zusätzlichen Aufwände, z.B. bei der Installation oder Eingewöhnung, abverlangen. Selbst auf Plattformen, die von \trademark{Microsoft Office} nicht offiziell unterstützt werden, wie z.B. Linux, existieren Programme mit denen das \trademark{Office}-Dokumenten-Format geöffnet und bearbeitet werden kann. Da \trademark{Office}-Dokumente in nur einer Datei gespeichert werden sind diese einfach auszutauschen -- in Agenturen werden die Dateien in der Regel auf einem Netzwerk-Laufwerk gespeichert, unternehmensfremde Mitarbeiter erhalten die Dateien via E-Mail, FTP-Server oder Filesharing-Anbieter. So wird das gemeinsame Arbeiten an den Texten, zumindest nacheinander, möglich. 

\secbar

Wie in diesem Abschnitt gezeigt wurde, sind Textverarbeitungs- und Tabellenkalkulationsprogramme wie \trademark{Microsoft} \trademark{Word} und \trademark{Excel} nominell für den Einsatz zur Verwaltung von Texten für Medienprodukte geeignet. Dies erklärt, warum sie zu Beginn eines Projekts als geeignet angesehen und in Agenturen immer wieder als Werkzeug für die Erfassung, Definition und Übersetzung der Texte eines Projekts ausgewählt werden. Im alltäglichen Gebrauch treten jedoch Probleme gerade im Bereich des gemeinsamen Bearbeitens, paralleler oder nachträglicher Änderungen und der Übertragung der fertigen Texte in den Produktionsprozess auf, die im folgenden Abschnitt erläutert werden.

\subsection{Probleme bei der Verwendung von Textverarbeitungs- und Tabellenkalkulationsprogrammen im Verlauf eines Projekts}
\label{l:officeprobleme}

Wie im vorangegangenen Abschnitt gezeigt wurde, sind Textverarbeitungs- und Tabellenkalkulationsprogramme wie \trademark{Microsoft} \trademark{Word} und \trademark{Excel} der Standard für die Verwaltung von Texten in Projekten zur Erstellung von In"-for"-ma"-ti"-ons- und Kom"-mu"-ni"-ka"-ti"-ons"-me"-di"-en. In den für diese Bachelor-Thesis geführten Interviews haben jedoch alle Personen von vielfältige Problemen in Zusammenhang mit diesen Werkzeugen berichtet. Dies belegt zum einen, dass im gängigen Workflow viele Möglichkeiten zur Verbesserung existieren und liefert zum anderen auch Hinweise, wie der verbesserte Workflow im Detail gestaltet werden muss. In diesem Abschnitt werden die beobachteten Probleme beschrieben.

\bigskip

\label{p:serielles-konzept}\paragraph{Serielles Bearbeitungskonzept} Das grundsätzliche Bearbeitungskonzept, das in \trademark{Word} und \trademark{Excel} zum Einsatz kommt, ist seriell, das bedeutet, dass ein Dokument gleichzeitig nur von einer Person bearbeitet werden kann. Soll mit mehreren Personen an einem Dokument gearbeitet werden, muss dieses zwischen allen Beteiligten ausgetauscht werden. Dies geschieht indem die Dokumenten-Datei entweder per E-Mail jeweils zum nächsten Bearbeiter verschickt wird oder die Datei auf einen, durch alle Beteiligten erreichbaren, Speicherort verschoben wird. In Agenturen handelt es sich hierbei meistens um eine Netzwerklaufwerk -- hierauf haben aber nur Mitarbeiter Zugriff, die Zugang zum lokalen Netzwerk der Agentur haben. Soll die Datei auch externen Mitarbeitern oder dem Kunden zur Verfügung gestellt werden, muss diese per E-Mail verschickt oder in extern erreichbare Speicherorte kopiert werden, wie z.B. FTP-Server, Wikis und Extranet-Portale. Üblich ist auch der Einsatz spezieller Programme zum Dateiaustausch, wie z.B. \trademark{Dropbox}. Die Organisation dieses Austausches ist besonders dann aufwändig, wenn Dateien sich nicht mehr unter Kontrolle der Agentur befinden, weil sie z.B. dem Kunden zur Abnahme geschickt wurden. Dann kommt es dazu, dass mehrere Versionen des Dokumentes parallel existieren: eine Version beim Kunden, die dort mit Änderungen und Ergänzungen versehen wird und eine Version in der Agentur dass sich aufgrund von Änderungen im Verlauf des Projekts ändert. Um anschließend alle Beteiligten auf den aktuellen Stand zu bringen, müssen die verschiedenen Versionen des Dokumentes manuell zusammengeführt werden -- automatisiert ist das mit \trademark{Word} und \trademark{Excel} nicht möglich. Neben dem zeitlichen Aufwand birgt das manuelle Zusammenführen weitere Fehlerquellen. Da Änderungen an den Texten durch Copy\&Paste übertragen werden, kann es gerade bei großen Dokumenten passieren, dass man die Änderungen an der falschen Stelle einarbeitet, sofern im Dokument sich ähnelnde Textabschnitte existieren, die leicht zu verwechseln sind. Aus Kostengründen und weil es sich dabei um eine repetitive Arbeit handelt ist es nicht selten der Fall, dass diese Änderungen von Praktikanten oder studentischen Aushilfen durchgeführt werden, die mangels inhaltlicher Kenntnis den Zusammenhang der Texte nicht kennen, was ebenfalls irrtümliche Änderungen an den Texten durch fehlerhaftes Copy\&Paste begünstigt. Auch ist die Eindeutigkeit der Dateiversionen nicht gewährleistet, allein aufgrund des Zeitstempels kann keine genaue Aussage darüber getroffen werden, welche Datei die neueste ist. So muss man sich auf ein Benamungsschema für Dateien einigen, das im besten Fall klar erkennen lässt \emph{welches} Dokument das neueste ist. Üblich sind dabei Ergänzungen des Dateinamens mit Datumsinformationen, wie z.B. \texttt{Text\_Online\_2012-04-13.docx}. Folgen nicht alle Beteiligten diesem Schema, weil sie z.B. nicht ausreichend informiert sind, oder kommt es zu gleichzeitiger Änderungen durch zwei Personen kann es so auch zu zwei verschiedenen Dateien mit dem gleichen Dateinamen kommen. Durch die Verteilung der Dokumente auf verschiedene Speicherorte kommt es zu Situationen, in denen nicht klar ist, wer aktuell die \emph{neueste} Dateiversion hat.

Tatsächlich existiert mit \trademark{Microsoft} \trademark{SharePoint}, einer Software für Intra-, Extra- und Internetportale, eine Lösung, die dieses Probleme behebt \cite{sharepoint-shared-documents}. Mit Hilfe von \trademark{Shared Documents} lassen sich Dokumente zentral ablegen. Sollen diese editiert werden, müssen Sie von der jeweiligen Person \typoquotes{ausgecheckt} werden, dies sperrt den Zugriff auf das Dokument durch andere Mitarbeiter. Sobald das Bearbeiten abgeschlossen wurde, wird die Datei durch den Bearbeiter wieder \typoquotes{eingecheckt}. So wird sichergestellt, dass es nie zwei Versionen der Datei mit unterschiedlichen Änderungen gibt. Diese Funktion behebt aber nicht den Umstand, das an einem Dokument immer nur eine Person gleichzeitig arbeiten kann. Desweiteren ist \trademark{SharePoint} im Agen"-tur-Um"-feld kaum anzutreffen, was dem Umstand geschuldet ist, dass der Betrieb einer \trademark{SharePoint}-Instanz mit hohen Lizenz- und Personalkosten verbunden ist. Zudem setzen die Funktionen zum gemeinsamen Bearbeiten von \trademark{Word}- oder \trademark{Excel}-Dokumenten voraus, dass alle Mitarbeiter über die neueste Programm-Version verfügen \cite{sharepoint-wordversions}. Dies kann aber nicht bei allen Projektbeteiligten, besonders nicht auf Kundenseite, vorausgesetzt werden.

\paragraph{Monolithische Dokumente} Das Zusammenführen aller Textbausteine eines Produkts in einem Dokument hat den Nachteil, dass dieses nur als Ganzes weitergeben werden kann. In bestimmten Konstellationen muss aber sichergestellt werden, dass bestimmte Inhalte nicht von allen Projektbeteiligten einsehbar sind. Zum einen kann es sich dabei um Informationen handeln, die sensibel sind oder der Geheimhaltung unterliegen, so dass sie nur bestimmten Personen zugänglich sein dürfen. Zum anderen kann es aus Kostengründen sinnvoll sein, die Prüfung von Texten durch Anwälte, oder die Übersetzung von Texten auf bestimmte Bereiche einzuschränken. In diesen Fällen wird es notwendig, verschieden Versionen des Dokumentes anzulegen, die an den jeweiligen Personenkreis angepasst sind. Dies erzeugt die Probleme, die im vorangegangenen Abschnitt beschrieben wurden.

\paragraph{Feedback} Durch das Verteilen der Dokumente auf verschiedene Speicherorte wird eine parallele Kommunikation des Arbeitsstandes mittels E-Mail nötig, bei der jeweils dem nächsten Bearbeiter mitgeteilt wird, dass er mit seiner Aufgabe weiter fortfahren kann. Der Ablauf und die Reihenfolge der Kommunikation ergibt sich durch die Aufgaben der beteiligten Personen, aber auch durch informelle Absprachen. Gerade zwischen Agentur und Kunden gibt es häufig \typoquotes{Flaschenhälse}, die zu Verzögerungen führen. Dies sind in den meisten Fällen die jeweiligen Projektleiter und Ansprechpartner, die auch bei technischen oder inhaltlichen Fragen jeweils der alleinige Empfänger sind, die Anfrage entgegen nehmen, in ihrem Unternehmen an die zuständige Person weiterleiten, auf deren Antwort warten und dann die Antwort zurück spielen. Hierdurch bilden sich umfangreiche und langlebige E-Mail-Kommunikationsketten, an denen viele, meistens zu viele, Personen beteiligt sind, die in ungeordneter Reihenfolge Feedback liefern.

\paragraph{Strukturierung von Dokumenten} Einer der Gründe, warum \trademark{Word} und \trademark{Excel} zur Standardausstattung auf jedem Büro-Computer gehören, ist, dass sie für einen sehr breiten Anwendungsbereich entwickelt wurden. Dies hat jedoch zur Folge, dass es mit einigem Aufwand verbunden ist, die passende Struktur für die Inhalte eines Produkts in einem Text- oder Tabellen-Dokument anzulegen. Hierbei wird meistens eine hierarchische Struktur mit Hilfe von Abschnitten angelegt (vgl. Abb. \ref{f:wordbooklet} · S.\pageref{f:wordbooklet}). In \trademark{Excel} wird im Hinblick auf die kompaktere Darstellung meistens auf besondere Formatierungen verzichtet und mit sich wiederholenden Zellen gearbeitet, die eine hierarchische Struktur simulieren (vgl. Abb. \ref{f:excelbooklet} · S.\pageref{f:excelbooklet}) -- die zweidimensionale Tabellendarstellung ist für komplexere Hierarchien nicht ausgelegt. Diese aufwändige Strukturierung des Dokuments muss auch geschehen, damit sich alle Anwender im Dokument zurecht finden können und eine eindeutige Zuordnung zwischen Textbausteinen im Dokument und den dafür vorgesehenen Platzhaltern im fertigen Produkt möglich ist. Da beide Programme für diese Aufgabe keinerlei Vorlage und Unterstützung liefern, muss hier viel Arbeit investiert werden, die zudem noch vorausschauend genug sein muss, um zu vermeiden, dass es im späteren Verlauf des Projekts durch nicht berücksichtige Fälle notwendig wird, das Dokument komplett zu überarbeiten.

\paragraph{Formatierung} Die Formatierung der Textbausteine nach gestalterischen Aspekten, also das Hinzufügen von z.B. Hervorhebungen, Unterstreichungen und Absätzen wird zum Teil schon während der Erstellung der Texte vorgenommen. Hierbei werden die jeweiligen Funktionen von \trademark{Word} und \trademark{Excel} verwendet. Ist dies in \trademark{Word} komfortabel möglich, sind die Möglichkeiten in \trademark{Excel} deutlich eingeschränkt. Hier lassen sich Zeichen-Formatierungen wie Hervorhebung, Farbe o.ä. nicht auf einzelne Worte oder Zeichen anwenden, sondern nur auf eine ganze Zelle. Auch Zeilenumbrüche stellen ein Problem dar. Diese sind zwar grundsätzlich möglich, jedoch kann es mangels Wissen dazu kommen, dass ein Bearbeiter einen Zeilenumbruch nicht innerhalb einer Zelle anlegt, sondern statt dessen eine neue Zeile einfügt. Werden die Zeilennummern oder bestimmte Spalten als Referenz-Schlüssel für den Text verwendet, führt das dazu, dass die zweite Zeile des Texts nicht mehr zugeordnet werden kann und im Produkt fehlt. 

Werden Tabellen in \trademark{Word}-Dokumenten verwendet um z.B. tabellarische Inhalte in einem Produkt zu beschreiben, führt das aufgrund der beschränkten Seitengröße eines Textdokuments dazu, dass Tabellen mit vielen Spalten nur mit sehr kleinem Text dargestellt werden können, was das Bearbeiten der Texte erschwert. 

Üblich ist auch das Einfügen von Bildern. Dies ist nötig, um die Zuordnung der Texte zu erleichtern oder um Untertitel für Fotos im Produkt zu definieren. Hierbei treten dann zusätzliche Formatierungsproblem auf, da das Platzieren von Bildern in \trademark{Office} nur in beschränktem Maße beeinflusst werden kann. 

Problematisch ist auch der Austausch der Formatierungen zwischen verschiedenen Programm-Versionen von \trademark{Word} und \trademark{Excel}, besonders wenn Dokumente von neueren Versionen in älteren Versionen angezeigt und bearbeitet werden und vor allem dann, wenn die Dokumente in anderen Textverarbeitungs- und Tabellenkalkulationsprogrammen wie z.B. \trademark{Apple} \trademark{iWorks} oder \trademark{LibreOffice} bearbeitet werden. Diese Programme unterstützen zwar \emph{per se} den Ex- und Import von \trademark{Microsoft}-Dateiformaten, bei der Konvertierung entstehen jedoch gerade beim Übertragen dieser Formatierungen Unsauberkeiten. 

Soll die Formatierungen in der Produktion dann schließlich in das Produkt übernommen werden, muss dies manuell geschehen, da sich diese Auszeichnungen nicht automatisch in die jeweiligen Produktionswerkzeuge übertragen lassen, weil dafür z.B. keine Konverter existieren oder die Formatierungsregeln nicht 1:1 übernommen werden können. Das manuelle Anlegen der Formatierungen im Produkt kann, ähnlich wie bei Copy\&Paste, zu Übertragungsfehlern führen.

\paragraph{Notizen und Anmerkungen} Neben den eigentlichen Textbausteinen werden in \trademark{Word}- und \trademark{Excel}-Dokumenten auch Hinweise und Kommentare zu den Texten mit Hilfe von Notizen oder direkt in das Dokument platzierten, besonders formatierten, Texten hinterlegt. Werden Notizen verwendet besteht zum einen das Problem, dass diese an einer spezifischen Stelle im Text platziert werden. Wird diese Stelle gelöscht, wird damit auch die Notiz ohne einen Warnhinweis gelöscht. Bei der gleichzeitigen Darstellung der Notizen in Kombination mit der Anzeige von Änderungen durch andere Bearbeiter kann das Dokument sehr unübersichtlich werden. Werden Hinweise als Text im Dokument hinterlegt kann dies dazu führen, dass diese Hinweise übersehen werden, oder bei Copy\&Paste von großen Abschnitten unbeabsichtigt in das Produkt übernommen  werden. Diese Unzulänglichkeiten führen dazu, dass Feedback auch parallel zu den Dokumenten ausgetauscht wird, meistens mit Hilfe von E-Mails in denen die Anmerkungen bzw. Änderungswünsche aufgezählt werden. Es muss dabei dann sichergestellt werden, dass die entsprechenden Personen über diese Informationen verfügen und auch wissen, welche gültig sind und welche veraltet.

Bei der Übersetzung von Texten kommt es mitunter vor, dass Hinweistexte, die z.B. Abschnitte kennzeichnen, übersetzt werden, da extern beauftragte Übersetzer sich mit den verwendeten Dokumentenvorlagen nicht auskennen. So werden dann auch beschreibende Texte wie \citequotes{Überschrift:} übersetzt, was zur Folge hat, dass das Dokument gänzlich unleserlich wird. Übersetzer erhalten auch oft nicht wichtige Zusatzinformationen zu Texten, wie z.B. Angaben über maximale Satzlängen, da sie die fertigen, abgenommenen Texte in der Ausgangssprache in einem separaten Dokument erhalten und bei der Erstellung des Dokuments diese, vermeintlich \emph{internen}, Hinweise oft nicht mit übernommen werden.

Informationen darüber, welche Teile zuletzt geändert wurden, können in \trademark{Word} und \trademark{Excel} zwar aufgezeichnet werden, diese Funktion muss aber explizit vom Überarbeiter aktiviert werden. Wird dies versäumt kommt es dazu, das Änderungen manuell durch Vergleichen von zwei Dokument-Versionen identifiziert werden müssen, wenn es nicht sinnvoll ist, alle Texte im Produkt noch einmal zu ersetzen.

\paragraph{Usabilty-, technische und typografische Probleme} Die bisher genannten Punkte beschreiben Probleme, die im Workflow entstehen, wenn \trademark{Word} und \trademark{Excel} eingesetzt werden. Aber auch in der Verwendung dieser Programme existieren weitere Probleme, die sich negativ auf die Arbeitsgeschwindigkeit auswirken. Zu den wichtigsten Usability-Problemen zählt, dass die Programme nicht für das gleichzeitige Arbeiten in mehreren Dokumenten ausgelegt sind, dies kommt besonders dann zum Tragen, wenn zwei Dokumente miteinander verglichen werden sollen; was z.B. beim Kontrollieren einer Übersetzung der Fall ist. Hierzu müssen das Originaldokument und die Übersetzung nebeneinander in zwei Fenstern geöffnet werden. Zum Vergleichen der Texte muss dann abwechselnd in diesen beiden Fenstern gescrollt werden. Kleinere Usability-Probleme sind etwa, dass das Anzeigen der Zeichenanzahl in einem Satz nur via Kontextmenü zu erreichen ist, diese Funktion aber häufig verwendet wird. 

Typografische Details wie optionale Wort-Trennungen oder optionale Zeilenumbrüche sind in \emph{Word} zwar möglich, können aber, wie die Formatierungen, nicht automatisch in das Endprodukt übernommen werden. Da Umbrüche sprachspezifischen Regeln folgen, müssen diese Informationen bereits während der Übersetzung hinterlegt werden, später in der Produktion werden sonst im Zweifelsfall die Umbrüche willkürlich gesetzt. 

Ein Problem technischer Art ist, dass sich bei großen Dokumenten mit oberen zweistelligen Seitenzahlen und mehr die Reaktionszeit der Anwendung merkbar verringert. Aufgrund unterschiedlicher Programm-Versionen aber auch unterschiedlicher Betriebssysteme kommt es beim Austausch der Dateien zu verschiedenen Kompatibilitätsproblemen, da in Agenturen meistens \trademark{Mac OS} verwendet wird, auf Kundenseite jedoch \trademark{Windows} verbreitet ist.

\paragraph{Workflow} Die Abwicklung des Workflows durch den Austausch von Dokumenten erfolgt nicht in einer geordneten Art und Weise. Die Reihenfolge, wer wann an den Dokumenten arbeitet ergibt sich organisch, da sich eine Reihenfolge nicht in den Dokumenten festlegen lässt. Soll ein gewisser Ablauf festgelegt werden, bei dem eine Vorbedingung erfüllt sein muss, bevor der nächste Mitarbeiter weiter arbeiten kann, muss dies parallel definiert und mit allen Beteiligten vereinbart werden. Die Kontrolle dieser Vereinbarungen liegt auch außerhalb der Dokumente und muss durch die einzelnen Mitarbeiter sichergestellt werden. Dieser Umstand, und vor allem die Tatsache, dass mündliche oder schriftliche Vereinbarungen in der Praxis immer wieder umgangen werden, führt letztendlich dazu, dass im Verlauf eines Projekts immer wieder Schleifen entstehen, die Mehrarbeit erzeugen.

\secbar

Dass \trademark{Word} und \trademark{Excel} trotz der genannten Probleme immer wieder zum Einsatz kommen, zeigt der nächste Abschnitt mit einer Reihe von Praxisbeispielen.

\subsection{Beispiele aus der Praxis}\label{l:praxisbeispiele}

Im Rahmen der für diese Bachelor-Thesis geführten Interviews mit Branchenexperten wurden die nachfolgenden Praxisbeispiele zusammengestellt. Diese zeigen für die in diesem Kapitel genannten Probleme Beispiele aus realen Projekten.

\subsubsection{Internetseite EnBW Transportnetze AG}

Bei \trademark{Scholz \& Volkmer} wurde im Rahmen der Auslagerung des Geschäftsbereiches Transportnetze der \trademark{EnBW Energie Baden-Württemberg AG} ein neues Informationskonzept für die Internetseite des neuen Unternehmens \trademark{EnBW Transportnetze AG}\footnote{\url{http://enbw-transportnetze.de/}} erarbeitet. Hierzu wurden die bestehenden Inhalte, die sich auf etwa 100 Seiten verteilten, analysiert und überarbeitet. Die überarbeiteten Texte wurden direkt in ein CMS übertragen und der neuen Struktur der Internetseite zugeordnet, die dann aus etwa 300 Einzel-Seiten bestand. Zur Abstimmung mit dem Kunden und den Fachabteilungen wurden aus dem CMS sogenannte \typoquotes{Content-Booklets} als \trademark{Word}-Dokument generiert (siehe Abbildung \ref{f:wordbooklet} · S.\pageref{f:wordbooklet}), die dann dem Kunden zur Abstimmung via E-Mail zur Verfügung gestellt wurden. Da die Freigabe des Booklets nicht auf einmal sondern nur Abschnittsweise erfolgte, mussten auch das externe Übersetzungsbüro nacheinander mehrere Dokumente übersetzen. Die freigegebenen und übersetzten Texte wurden dann von studentischen Aushilfen wieder mit Copy\&Paste im CMS korrigiert bzw. eingetragen. Die größten Probleme in diesem Projekt war der zusätzliche Arbeitsaufwand, um die \trademark{Word}-Dokumente zu erzeugen, der Zeitaufwand beim Einpflegen der neuen Texte und Übersetzungen und die Verwaltung der abgenommenen und nicht abgenommen Teile der Texte.

\subsubsection{Banner-Kampagne Nintendo}

Für ein Computerspiel lässt \trademark{Nintendo} Flash-Werbebanner in verschiedenen Formaten, mit unterschiedlichen Motiven und Sprachvarianten anfertigen. Die mit der Umsetzung beauftragten Mitarbeiter bekommen hierfür die Texte in Word-Dokumenten geliefert und müssen diese mittels Copy\&Paste übertragen. Hierbei muss darauf geachtet werden, dass die jeweiligen Motive mit den passenden Texten versehen werden, entsprechend umfangreich sind die Hinweise im \trademark{Word}-Dokument. Das größte Problem sind die fehlenden Hinweise, wie man Texte korrekt umbricht, da es gerade bei Werbebannern Texte mit nur wenigen Worten pro Zeile gibt. Diese Zusatzinformationen werden aber oft nicht vom Übersetzer hinterlegt.

\subsubsection{EA Phenomic: BattleForge}

Mehrere tausend Texte und deren Übersetzung für das auf Spielkarten basierende Echtzeitstrategiespiel \trademark{BattleForge} von \trademark{EA Phenomic} wurde bei diesem Projekt mit Hilfe einer \trademark{Excel}-Tabelle verwaltet. Hierbei wurden in der ersten Spalte Identifier für alle Texte vergeben. In den weiteren Spalten wurden die Übersetzungen eingetragen. Übersetzungsbüros haben die Texte mit Hilfe einer Kopie der Datei übersetzt. Praktikanten haben die übersetzten Texte dann wieder in die Master-Datei eingepflegt. Hierbei kam es regelmäßig zu Fehlern, zum einen wurden Zeilen und damit Identifier gelöscht und zum anderen wurden mehrzeilige Texte manchmal versehentlich in mehrere Zeilen geschrieben. Dabei wurden neue Zeilen eingefügt, die dann aber keinen Identifier mehr besaßen. Parallel zu den Arbeiten des Übersetzungsbüros kam es aber auch immer wieder zu Änderungen an den Ursprungstexten und Identifiern. Diese Änderungen musst dann in allen Sprachversionen nachträglich überarbeitet werden. Zusätzlich wurden die Texte durch Markenanwälte begutachtet, auch deren Feedback musste wieder in das \trademark{Excel}-Dokument eingepflegt werden und hatte ggfs. Einfluss auf die Texte im Spiel und die Übersetzung.

\subsubsection{MAN Truck \& Bus AG: Neufahrzeug-Konfigurator}

Im Neufahrzeug-Konfigurator der \trademark{MAN Truck \& Bus AG} existieren über 20.000 Texte in 18 Sprachen. Diese Texte werden zentral in einem CMS verwaltet. Müssen neue Texte eingepflegt oder aktualisiert werden erhalten die Landesniederlassungen ein Delta als \trademark{Excel}-Dokument, das in der Regel alle vier Monate jeweils 1.000 bis 1.500 geänderte Texte enthält. Nachdem die Texte von der Landesniederlassung übersetzt wurden, werden diese von einer Übersetzungsagentur geprüft und mittels Copy\&Paste wieder in das CMS eingepflegt.  Probleme entstehen hier, wenn sich Landesniederlassungen nicht an das Dokumenten-Format halten und zusätzliche Texte, z.B. Übersetzungsalternativen, einfügen. Der Versand der Dokumente erfolgt durch einen zentralen Datenverantwortlichen, der sicherstellen muss, dass alle Landesniederlassungen ihre Dokumente bekommen, wieder zurücksenden und sie dem Übersetzungsbüro anschließend zur Verfügung gestellt werden.

\subsection{Schlussfolgerung}\label{l:schlussfolgerung}

In diesem Kapitel wurde erläutert, warum Text eine besondere Rolle bei der Erstellung von In"-for"-ma"-ti"-ons- und Kom"-mu"-ni"-ka"-ti"-ons"-me"-di"-en"-pro"-dukt"-en spielt. Es wurde gezeigt, dass \trademark{Microsoft Word} und \trademark{Excel} ausgewählt werden, um die Texte für diese Produkte zu Verwalten und den vielen Projektbeteiligten zugänglich zu machen. Der Grund warum diese Programme für die Erfassung, Beschreibung, Erstellung, Übersetzung und Übertragung von Texten in das Produkt verwendet werden ist der, dass keine dedizierten Lösungen existieren, die explizit die genannten Abläufe abbilden. Statt dessen wird Software verwendet, die bei allen Beteiligten vorhanden ist und mit denen diese bereits vertraut sind, wodurch sie mit den Texten in einem gewohnten Umfeld arbeiten können. Die Verwendung von Dateien ermöglicht den Austausch unter den Beteiligten im Netzwerk, per E-Mail oder über Programme. In diesem Kapitel wurde jedoch gezeigt, dass diese Wahl mit vielen Nachteilen verbunden ist und im Projektverlauf viele Stellen eröffnet, an denen es zu Problemen kommt oder es wegen der schlechten Eignung zu einer langsamen Arbeitsgeschwindigkeit führt. Die Probleme dieses Workflows werden jedoch erst im Verlauf des Projekts sichtbar und betreffen vor allem Agenturen, die als Dienstleister in einem Abhängigkeitsverhältnis stehen. Auf deren Seite werden die Missstände durch Mehrarbeit und aufwändige, sich wiederholende Arbeitsschritte ausgeglichen, die aufgrund ihrer Natur fehleranfällig sind.

\secbar

Dieses Kapitel zeigt, dass vorhandene Werkzeuge entweder ungeeignet sind, oder mangels Akzeptanz oder Komplexität nicht eingesetzt werden. Ziel dieser Bachelor-Thesis ist es, eine Anwendung zu konzipieren, die die genannten Probleme beseitigt. Für die Akzeptanz einer solchen Anwendung ist es besonders wichtig, dass sich diese stark an den Bedürfnissen ihrer Benutzer orientiert und für die unterschiedlichen Nutzungsarten und Erfahrung der Benutzer im Umgang mit Softwaresystemen den passenden Zugang bietet. Als Grundlage für die Konzeption der Anwendung in Kapitel \ref{l:konzeption} · S.\pageref{l:konzeption} werden im nächsten Kapitel Personas vorgestellt, die die typischen Nutzer dieser Anwendung repräsentieren.

\pagebreak

\section{Konzeption eines an die spezifischen Probleme angepassten Workflows}

\subsection{Vorraussetzung / Abgrenzung}

Content-Management-Systeme bzw. Redaktionssystem können einen Teil der Aufgabe abbilden, sind aber i.d.R. ungeeignet (z.B. kein Workflow), keine Context-Informationen hinterlegbar.

\subsection{Workflow}

Beschreibung des optimalen Workflows und die Rolle der Beteiligten

Innerhalb der Anwendung wird das Projekt angelegt und die dafür benötigten Textbausteine definiert. Hierbei können detaillierte Angaben zu deren Eigenschaften gemacht werden, z.B. über den Verwendungszweck oder die maximal Länge. Die einzelnen Textbausteine werden bei diesem Vorgang entsprechend dem Aufbau des Endproduktes in eine Reihenfolge gebracht und hierarchisch angeordnet. So wird eine leichte Orientierung und Zuordnung der Text zum Endprodukt möglich. 

Nachdem die benötigten Textbausteine definiert wurden, werden diese durch Texter befüllt. Für Texter stellt die Anwendung Hilfsfunktionen zur Verfügung. Dazu zählen Informationen wie Zeichenlänge und Wortanzahl und Rechtschreibkorrektur mit Wörterbuch.

Sobald die Texte hinterlegt wurden durchlaufen sie die Qualitätskontrolle durch andere Mitarbeiter des Projektes und anschließend den Freigabeprozess beim Kunden. Wurden die Texte freigegeben, können die zusammengestellten Texte in das Endprodukt übernommen werden. 

Alle Vorgänge werden innerhalb der Anwendung protokolliert und sind so für jeden Beteiligten leicht nachvollziehbar. Aufgaben können automatisch aufgrund von Änderungen erzeugt werden, oder von Mitarbeiter angelegt werden. So wird sichergestellt, dass alle Projektmitarbeiter jederzeit über ihre Aufgaben bezüglich der Texte informiert sind, bei Änderungen die verantwortlichen Mitarbeiter informiert werden. Dadurch wird es möglich auch bei Korrekturen in letzter Minute diese Änderungen gezielt und transparent zu übernehmen.

\subsection{Beschreibung der notwendigen Funktionalität}

Unterteilung in Muss- und Kann-Kriterien

\subsection{Nachteile/Risiken des Konzepts}

\subsection{Personas}

Vorstellung (basierend auf Interviews mit realen Personen), Analyse des Konzepts in Bezug auf Personas

\subsubsection{Texter}

Copywriter sind die eigentlichen Texter, die lediglich Text erstellen, für die kein Fachwissen nötig ist, oder dieses schon vorliegt. Copywriter können Spezialwissen bezüglich SEO haben. Redakteure sind für die Gesamtheit der Texte verantwortlich und stellen sicher, dass globale Vorgaben erfüllt werden. Journalisten erstellen Texte, die auf Recherchen basieren. Diese Texte unterliegen der Sorgfaltspflicht, Quellen müssen nicht genannt werden. 

\section{Entwurf einer Anwendung}

Diese vier Leitlinien repräsentieren die Grundgedanken bei der Entwicklung von der Anwendung:

\begin{itemize}
\item{Das wichtigste zuerst: Die aktuelle Aufgabe soll immer im Fokus der Darstellung liegen.}
\item{Schnell zum Ziel: Alle Aufgaben müssen leicht und umkompliziert durchführbar sein.}
\item{Nicht nerven: Ständige Benachrichtigungen lenken ab und müssen deswegen so gestaltet sein, dass diese sich nach den Präferenzen des Nutzers richten.}
\item{Hilfe nur einen Klick entfernt: Das Hilfesystem muss kontextsensitiv verfügbar sein und ist eine Kernfunktion der Anwendung}
\end{itemize}

\subsection{Überblick}

Diese Abbildung liefert einen Überblick über den Aufbau des Systems:

\begin{figure}[htb]
\begin{center}
\includegraphics[width=\textwidth]{media/System.pdf}
\caption{Aufbau des Systems}
\label{chart:1}
\end{center}
\end{figure}

Die Zentrale Komponente der Anwendung bildet der Server. Für die Benutzer erfolgt der Zugriff mit Hilfe einer GUI, die mit der REST-API des Servers kommuniziert. In der ersten Version wird eine browserbasierte GUI auf Basis von HTML5 und JavaScript existieren, die auch schon auf Smartphones verwendet werden kann. Später kommen dann spezielle Plugins für Adobe-Produkte und weitere wichtige Produktionsumgebungen hinzu. Auch native GUIs für Smartphones verwenden die gleiche API. Die Schnittstellen können auch von Drittanbietern dazu verwendet werden, eigenen Clients für das System zu entwickeln. In die Endprodukte gelangen die Texten über den Export, exportiert wird dabei in viele Formate, neben Datenformaten wie z.B. XML werden auch Dokumentenformate wie z.B. Word exportiert. Der Export kann durch den Anwender erzeugt werden (\emph{Pull-Export}), aber auch automatisch, z.B. nach festgelegten Zeitplänen oder Ereignissen erfolgen. Dieser \emph{Push-Export} erfolgt auf je nach Projekt festlegbaren Orte, wie z.B. FTP-Server oder Versionsverwaltungssysteme. Die Benachrichtigungen über Aufgaben und Änderungen an Texten kann via E-Mail, aber auch mittels Instant-Messaging-Systeme oder durch den Aufruf fremde API-Endpunkte erfolgen – dies ist ebenfalls innerhalb eines Projektes und pro Nutzer individuell konfigurierbar.

\subsection{Schnittstellen}

Anforderungen, Umfang, Ausprägung für Import-, Export- und Benachrichtigungsschnittstellen

Anbindung via CMIS http://en.wikipedia.org/wiki/Content\_Management\_Interoperability\_Services

Export eines Text-Booklets für die Rechtschreibkontrolle. Identifier mit ausgeben, um Texte dann schnell finden zu können. Hier könnte man auch einen QR-Code drucken, dann kann man mit einer mobilen App den Text direkt ändern.

\subsection{Grundüberlegung zu einer GUI}

Anforderungen, Grundsätze, Usability, Aufbau, Wireframes

Bei Kontroll-Aufgaben (Lektorat, QS) unterbrechungsfreies Arbeiten ermöglichen (Infinite-Scroll).

\section{Implementierung eines Prototypen}\label{l:implementierung}

\subsection{Grundüberlegung}

Content-Management-Systeme bzw. Redaktionssystem können einen Teil der Aufgabe abbilden, sind aber i.d.R. ungeeignet (z.B. kein Workflow), keine Context-Informationen hinterlegbar.

\subsection{Abgrenzung}
\subsection{Beschreibung der gewählten Umsetzung, Komponenten}

\subsection{Anwendung der Umsetzung am Beispiel des Studiengangsflyers}

Die vorgeschlagene Lösung wird anhand eines realen Projektgs auf ihre praxistauglichkeit hin überprüft. Es handelt sich dabei um die einmal im Jahr erscheinende Informationsbröschüre des Studienganges Medieninformatik an der Hochschule RheinMain. Die Bröschure zu Beginn des Wintersemesters 2011/2012 einen Umfang von 28 Seiten zuzüglich Titel und Rückseite. In ihr findet sich das Grußwort des Studiengangsleiters, eine Kurzinfo über den Studiengang, das Studienprogramm mit Informationen zum Verlauf des Studiums, ein Terminkalender, Informationen zu Einrichtungen des Fachbereiches, eine Liste mit Personen im Fachbereich, sowie eine Umgebungskarte und ein Gebäudeplan. Die Broschüre wird von den Mitarbeitern des Fachbereiches selber erstellt.

\paragraph{Workflow}

Innerhalb der Anwendung wird das Projekt angelegt und die dafür benötigten Textbausteine definiert. Hierbei können detaillierte Angaben zu deren Eigenschaften gemacht werden, z.B. über den Verwendungszweck oder die maximal Länge. Die einzelnen Textbausteine werden bei diesem Vorgang entsprechend dem Aufbau des Endproduktes in eine Reihenfolge gebracht und hierarchisch angeordnet. So wird eine leichte Orientierung und Zuordnung der Text zum Endprodukt möglich. 

Nachdem die benötigten Textbausteine definiert wurden, werden diese durch Texter befüllt. Für Texter stellt die Anwendung Hilfsfunktionen zur Verfügung. Dazu zählen Informationen wie Zeichenlänge und Wortanzahl und Rechtschreibkorrektur mit Wörterbuch.

Sobald die Texte hinterlegt wurden durchlaufen sie die Qualitätskontrolle durch andere Mitarbeiter des Projektes und anschließend den Freigabeprozess beim Kunden. Wurden die Texte freigegeben, können die zusammengestellten Texte in das Endprodukt übernommen werden. 

Alle Vorgänge werden innerhalb der Anwendung protokolliert und sind so für jeden Beteiligten leicht nachvollziehbar. Aufgaben können automatisch aufgrund von Änderungen erzeugt werden, oder von Mitarbeiter angelegt werden. So wird sichergestellt, dass alle Projektmitarbeiter jederzeit über ihre Aufgaben bezüglich der Texte informiert sind, bei Änderungen die verantwortlichen Mitarbeiter informiert werden. Dadurch wird es möglich auch bei Korrekturen in letzter Minute diese Änderungen gezielt und transparent zu übernehmen.

\pagebreak

\section{Fazit}\label{l:fazit}

Diese Bachelor-Thesis hat sich mit der Fragestellung beschäftigt, wie es dazu kommt, dass trotz aller technischen Fortschritte im Bereich der Informationstechnologie bei der Organisation von Texten für Medienprodukte auf Arbeitsweisen zurückgegriffen wird, die inzwischen überholt sein sollten und wie eine bessere Lösung für diese Aufgaben aussehen könnte.

Es wurde gezeigt, dass die gebräuchlichen Werkzeuge, \trademark{Microsoft Word} und \trademark{Excel}, in der alltäglichen Arbeit in Agenturen zu vielerlei Problemen führen, sie aber verwendet werden, weil die Nutzer zum einen deren Gebrauch gewöhnt sind und zum anderen die Werkzeuge scheinbar über alle notwendigen Funktionen für diese Aufgabe verfügen. In einer ausführlichen Analyse wurde diese Annahme jedoch widerlegt und im Einzelnen gezeigt, welche problematischen Auswirkungen der Einsatz monolithischer Dateiformate und dezentraler Speicherung in den komplexen Abläufen in Zusammenhang mit der Erstellung von Medienprodukten haben.

Aufbauend auf dieser Erkenntnis und unter Zuhilfenahme von Personas, die auf Interviews mit zwölf Branchenexperten basieren, wurde eine Lösung konzipiert, die versucht, die genannten Probleme zu beseitigen und den Anforderungen der Personas zu genügen. Hierzu wurde ein zentraler Anwendungsserver vorgeschlagen, auf den mit spezialisierten, an die jeweiligen Bedürfnisse der Benutzer angepassten, GUIs zugegriffen wird. Mit deren Hilfe werden die Texte der Produkte definiert, geschreiben, korrigiert, kontrolliert, freigegeben und veröffentlicht.

Für die wichtigsten Bestandteile der Lösung, den Anwendungsserver und das browserbasierte GUI, wurde die konkrete Architektur entworfen und detaillierte Gestaltungsrichtlinen mithilfe von Wireframes festgelegt.

Zur Validierung des Entwurfs wurde schließlich ein Prototyp umgesetzt, der die wichtigsten Funktionen anhand eines Beispiel-Projekts implementiert. Die Implementierung zeigte, dass das Konzept funktioniert, der Entwurf realisierbar ist und bereits die prototypische Fassung konkreten Mehrwert bietet.

\secbar

Diese Bachelor-Thesis liefert eine konkrete Empfehlung für die Realisierung einer Anwendung zur Verwalten von Texten für Medienprodukte. Sie orientiert sich dabei an den tatsächlichen Abläufen in Projekten zur Erstellung von Informations- und Kommunikations-Medien und den Bedürfnissen der beteiligten Personen. Die vorgestellte Lösung bietet die Möglichkeit, im Projektverlauf in großem Maße Zeit einzusparen und Fehler zu vermeiden.

\pagebreak

\pagebreak

\bibliography{parts/bibliothek}

\end{document}