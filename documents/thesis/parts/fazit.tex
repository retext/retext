\section{Fazit}\label{l:fazit}

Diese Bachelor-Thesis hat sich mit der Fragestellung beschäftigt, wie es dazu kommt, dass trotz aller technischen Fortschritte im Bereich der Informationstechnologie bei der Organisation von Texten für Medienprodukte auf Arbeitsweisen zurückgegriffen wird, die inzwischen überholt sein sollten und wie eine bessere Lösung für diese Aufgaben aussehen könnte.

Es wurde gezeigt, dass die gebräuchlichen Werkzeuge, \trademark{Microsoft Word} und \trademark{Excel}, in der alltäglichen Arbeit in Agenturen zu vielerlei Problemen führen, sie aber verwendet werden, weil die Nutzer zum einen deren Gebrauch gewöhnt sind und zum anderen die Werkzeuge scheinbar über alle notwendigen Funktionen für diese Aufgabe verfügen. In einer ausführlichen Analyse wurde diese Annahme jedoch widerlegt und im Einzelnen gezeigt, welche problematischen Auswirkungen der Einsatz monolithischer Dateiformate und dezentraler Speicherung in den komplexen Abläufen in Zusammenhang mit der Erstellung von Medienprodukten haben.

Aufbauend auf dieser Erkenntnis und unter Zuhilfenahme von Personas, die auf Interviews mit zwölf Branchenexperten basieren, wurde eine Lösung konzipiert, die versucht, die genannten Probleme zu beseitigen und den Anforderungen der Personas zu genügen. Hierzu wurde ein zentraler Anwendungsserver vorgeschlagen, auf den mit spezialisierten, an die jeweiligen Bedürfnisse der Benutzer angepassten, GUIs zugegriffen wird. Mit deren Hilfe werden die Texte der Produkte definiert, geschreiben, korrigiert, kontrolliert, freigegeben und veröffentlicht.

Für die wichtigsten Bestandteile der Lösung, den Anwendungsserver und das browserbasierte GUI, wurde die konkrete Architektur entworfen und detaillierte Gestaltungsrichtlinen mithilfe von Wireframes festgelegt.

Zur Validierung des Entwurfs wurde schließlich ein Prototyp umgesetzt, der die wichtigsten Funktionen anhand eines Beispiel-Projekts implementiert. Die Implementierung zeigte, dass das Konzept funktioniert, der Entwurf realisierbar ist und bereits die prototypische Fassung konkreten Mehrwert bietet.

\secbar

Diese Bachelor-Thesis liefert eine konkrete Empfehlung für die Realisierung einer Anwendung zur Verwalten von Texten für Medienprodukte. Sie orientiert sich dabei an den tatsächlichen Abläufen in Projekten zur Erstellung von Informations- und Kommunikations-Medien und den Bedürfnissen der beteiligten Personen. Die vorgestellte Lösung bietet die Möglichkeit, im Projektverlauf in großem Maße Zeit einzusparen und Fehler zu vermeiden.

\pagebreak