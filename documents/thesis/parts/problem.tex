\section{Problem-Analyse}

In diesem Abschnitt wird das Problem bei der Verwaltung von Texten in Medien beschrieben. Nach der Definition der in dieser Bachelor-Thesis verwendeten Begriffe werden die typischen Probleme, die bei der Verwaltung von Texten für Medien auftreten beschrieben und mit Beispielen aus der Praxis belegt.

\subsection{Definition}
\label{l:def}

In dieser Bachelor-Thesis werden bestimmte allgemeine Begriffe und deren Synonyme verwendet, deren konkrete Bedeutung im Kontext dieser Arbeit wie folgt definiert ist:

\paragraph{Workflow (Ablauf)} Die Automatisierung eines Business-Prozesses, als Ganzes oder in Teilen, in welchem Dokumente, Informationen oder Aufgaben entsprechend einer Menge von prozeduralen Regeln von einem zum anderen Teilnehmer zur Bearbeitung weitergegeben werden \cite[S.8]{wmc}. Allgemein lässt sich sagen, dass ein Workflow aus den zum Erreichen eines Ziels nötigen Arbeitsschritten besteht.

\paragraph{Medium (Medien-Produkt, Produkt)} Medien sind physische oder elektronischer Informationsträger. Diese Bachelor-Thesis beschäftigt sich vor allem mit \ac{IKTM}, und hierbei vor allem Massenkommunikationsmittel, sowohl physischer als auch technischer Natur (vgl. \cite[S.199–201]{schanze2002metzler}). Dies können z.B. Marketingmedien wie Broschüren oder Fernsehwerbespots sein aber auch Software wie eine Smartphone-App oder eine Internetseite. 

\paragraph{Text (Textbaustein)} Damit sind die kleinsten sinnvoll identifizierbaren Bestandteile gemeint, aus denen sich der Text eines Produktes zusammensetzt. Dies sind in der Regel einzelne Sätze bei Druckmedien, können aber auch einzelne Worte sein, wie z.B. die Beschriftung einer Schaltfläche in einer Anwendung.

\paragraph{Agentur} Ein Unternehmen das Medien erstellt. In der Regel sind dies Werbeagenturen, Medien-Produktionsfirmen oder Software-Systemhäuser. 

\paragraph{Projekt} Die Erstellung von Medien erfolgt innerhalb von Agenturen in Projektarbeit. Projekte sind zeitlich begrenzt und vereinen zielgerichtet die zur Erstellung des Produktes beteiligten Mitarbeiter und Ressourcen. 

\paragraph{Kunde} Ein Unternehmen das Agenturen mit der Erstellung von Medien beauftragen.

\subsection{Die besondere Rolle von Text in \acl{IKTM}}

Es existieren nahezu keine Medien, die ohne Texte auskommen, denn Text ist im Gegensatz zu Grafiken, Fotos oder Animationen ein eindeutiger Informationsträger und unterliegt viel weniger stark einer Interpretation durch den Rezipienten eines Mediums als die symbolisierte oder stilisierte Darstellung von Informationen in audiovisuellen Medien. Text wird in der Marketing-Kommunikation als Unterstützung der zu übermittelnden Information verwendet. Hat man die Aufmerksamkeit des Betrachter eines Produkts erlangt, liefert Text weitere Informationen zum Produkt, er dient dazu, die emotionale Botschaft zu erläutern und zu präzisieren. Auch aus rechtlichen Aspekten ist Text aus den genannten Gründen der einzige verbindliche Informationsträger – bestes Beispiel hierfür ist das sogenannte \typoquotes{Kleingedruckte}, dass sich gerade bei inhaltlich sehr stark komprimierten Werbeformen, wie z.B. Plakat- oder Fernsehwerbung, findet. Ist die Textmenge, die in der Marketing-Kommunikation zum Einsatz kommt, noch überschaubar, gibt es doch Medien, deren Hauptbestandteil Text ist. Hierunter fallen klassische Druckerzeugnisse wie Broschüren und Kataloge oder Produkte der Unternehmenskommunikation wie Jahresberichte und Pressemeldungen. Besonders digitale Medien werden oft mit großen Textmengen versehen – von der einfachen Produkt-Microsite, über Werbemittel wie Newsletter bis zur Unternehmenswebsite – die Möglichkeit Inhalte hierarchisch zu strukturieren und sogar über eine Suche zugänglich zu machen hebt eine Limitierung des Umfanges, wie bei Druckprodukten, praktisch auf.

\begin{figure}[htb]
\begin{center}
\includegraphics[width=\textwidth]{media/chart-2.pdf}
\end{center}
\caption{Bei der Erstellung von Texten für \ac{IKTM} beteiligte Personen}
\label{chart:2}
\end{figure}

Betrachtet man die Abläufe von Projekten, in deren Verlauf Medien erstellt werden, lassen sich bezüglich der Textbestandteile dieser Produkte immer wieder sehr ähnliche Vorgehensweisen und Besonderheiten beobachten. Aufgrund der verbindlichen Natur von Text sind an der Erstellung der Texte für ein Medium mehr Personen beteiligt, als es z.B. für die Gestaltung des Produkts, der Auswahl von Bildmaterial oder für die Programmierung der Fall ist,  da er sehr viele verschieden Kriterien erfüllen muss. Tabelle~\ref{table:textkriterien} auf Seite~\pageref{table:textkriterien} listet exemplarisch eine typische Gruppe von Personen auf, die im Verlauf eines Projektes Einfluss auf den Text eines Produktes haben. Dieser Einfluss wird dabei in der Regel nicht in einer sinnvollen Reihenfolge und im Sinne eines geplanten Projektverlaufes ausgeübt, Hinweise von Anwälten sollten im besten Fall vor einer Übersetzung vorliegen, sondern richtet sich nach den Terminplänen der Personen. Gerade auf die Kundenseite haben Agenturen keinen Einfluss. In Projektplänen lassen sich zwar verbindliche Termine für die Lieferung von Texten des Kunden festlegen, dies verhindert aber keinesfalls, dass zu einem späteren Zeitpunkt Änderungen durchgeführt werden. Auch die Art der Kriterien sind sehr vielfältig: Im Entwurf und in der Umsetzung der Produkte legen Designer, Architekten und Produzenten die Struktur von Text wie Art der Ansprache, maximale Wortlänge, Anzahl der Wörter einer Überschrift fest oder diese werden durch das verwendete Medium vorgegeben, Texter erstellen die die Inhalte fest, die wiederum durch Wünsche des Kunden beeinflusst werden; das Lektorat, Fachabteilungen und Anwälte begutachten die Texte dann bezüglich der jeweils erforderlichen Korrektheit.

\begin{table}
\begin{center}
\begin{tabular}{@{}l l l l}
\textbf{Kriterium} & \textbf{Art} & \textbf{Verantwortlich} & \textbf{Organisation}\\
\hline
Aufgabenverteilung & Mitarbeiter & Projektleiter & Agentur\\
\hline
Zielgruppe & Struktur & Informationsarchitektur & Agentur\\
\hline
Umfang, Satzlänge & Struktur & Art-Direktion & Agentur\\
\hline
Länge einzelner Wörter & Struktur & Programmierer & Agentur\\
\hline
Information & Inhalt & Texter & Extern\\
\hline
Orthographie & Korrektheit & Lektorat & Extern\\
\hline
Übersetzung & Sprache & Übersetzungsbüro & Extern\\
\hline
Suchmaschinen-Optimierung & Inhalt & SEO-Experte & Extern\\
\hline
Aufgabenverteilung & Mitarbeiter & Projektleiter & Kunde\\
\hline
Fachliche Aspekte & Korrektheit & Fachabteilung & Kunde\\
\hline
Rechtliche Aspekte & Korrektheit & Rechtsanwalt & Kunde\\
\hline
Werbeaussagen & Inhalt & Marketingabteilung & Kunde\\
\hline
… & … & …
\end{tabular}
\caption{Kriteren von Textbausteinen und Verantwortliche}
\label{table:textkriterien}
\end{center}
\end{table}

Wie man Tabelle~\ref{table:textkriterien} entnehmen kann, existieren vielfältige Einflussmöglichkeiten auf die Gestaltung von Texten für Medien die sich auf viele Verantwortliche verteilen. Der Grund dafür ist, dass alle Beteiligten jeweils spezifisches Fachwissen in den Text einfließen lassen, seien es gestalterische Aspekte, die Einfluss auf die Struktur haben, oder das wissen über exakte technische Abläufe, die nur Spezialisten in den Fachabteilungen auf Kundenseite bekannt sind. Dieses Expertenwissen kann nicht für die meist kurze Projektlaufzeit an die umsetzenden Agentur vermittelt werden. Es ist also unvermeidlich, dass Text während des gesamten Projektverlauf geändert werden kann. Neben den Einflüssen durch Experten gibt es auch projektbedingte Einflüsse auf Text in letzter Minute. Sind in Texten Informationen enthalten sind, die einen zeitlichen Aspekt abbilden, ergeben sich durch Verzögerungen im Projekt automatisch Änderungsanforderungen. Ein Beispiel sind Gewinnspiele: Verschieben sich durch Probleme während dem Projekt die Zeiten, ab wann ein Produkt beim Rezipienten vorliegt, müssen auch eventuell knapp kalkulierte Gewinnspieltermine angepasst werden. Ein weiterer Grund für vielfältige Textänderungen im Verlauf eines Projektes ist die Erwartungshaltung des Kunden -- da es Kunden aus ihrem eigenen Arbeitsalltag gewöhnt sind, mit Textverarbeitungsprogramme zu arbeiten, und sie so aus eigener Erfahrung vermeintlich wissen dass Texte schnell geändert sind, erwarten sie auch, dass die Texte im Produkt bis zum Schluss geändert werden können; ihnen ist nicht bewusst, das vom ursprünglichen Satz bis zur Darstellung im fertigen Produkt durchaus viele aufwändige Arbeitsschritte nötig sein können.

\bigskip

In diesem Abschnitt wurde gezeigt, das Texte in Medien durch viele Personen und über den gesamten Verlauf eines Projektes geändert werden können. Im nächsten Abschnitt wird erläutert, wie der Austausch über die Textänderungen erfolgt und welche Probleme dabei entstehen.

% MARK

\subsection{Das Werkzeug der Wahl zur Verwaltung von Text: \emph{Microsoft Office}}

So komplex auch die Abläufe bei der Erstellung von Texten sind, um so erstaunlicher ist die Tatsache, dass das Werkzeug der Wahl zur Abbildung dieser Prozesse in den allermeisten Fällen \emph{Microsoft Word} oder \emph{Excel} ist. Auf den ersten Blick bilden diese Werkzeuge viele der benötigten Funktionen rund um die Textprozesse ab, aber im alltäglichen Gebrauch treten viele Probleme gerade im Bereich des gemeinsamen Bearbeitens, paralleler oder nachträglicher Änderungen und der Übertragung der fertigen Texte in den Produktionsprozess auf. Der Grund für die Wahl der \emph{Office}-Produkte liegt auf der Hand: sind sie doch in den allermeisten Unternehmen der Standard zur Textverarbeitung und sogar plattformunabhängig verfügbar – zumindest existiert die Möglichkeit das Microsoft Office-Dateiformat auf allen Platformen zu bearbeiten. Da bei allen Projektbeteiligten eine Installation von \emph{Microsoft Office} vorausgesetzt werden kann, werden sie zu \typoquotes{leichtgewichtige} Werkzeugen, die vom Anwender keine zusätzlichen Aufwände z.B. bei der Installation oder Eingewöhnung erfordern. Selbst auf Plattformen, die von \emph{Microsoft Office} nicht offiziell unterstützt werden (z.B. Linux) existieren Programme mit denen das Office-Dokumenten-Format geöffnet und bearbeitet werden kann.

\subsection{Die verwendeten Office-Funktionen}

Als klassisches Textverarbeitungsprogramm verfügen Office-Programme über viele Funktionen, die die Erstellung von Texten erleichtern.

\begin{itemize}
\item{Rechtschreibkorrektur für alle üblichen Sprachen}
\item{Kommentarfunktion}
\item{Änderungsfunktionen (Nachverfolgen, wer was geändert hat)}
\item{Möglichkeit zur hierarchischen Strukturierung der Texte in Seiten, Kapitel und Abschnitte}
\item{Möglichkeit zum Anlegen eines Inhaltsverzeichnisses (in Word)}
\item{die tabellarische Ansicht in Excel ermöglicht eine übersichtliche Darstellung, meist mit der Originalsprache in der ganz linken Spalte, pro Zeile ein Text, die Übersetzungen dann in den weiteren Spalten}
\item{Export-Funktion nach PDF}
\item{Globales Suchen und Ersetzen}
\item{Formatierungsfunktionen (fett, kursiv, farblich) zum Hervorheben von wichtigen Passagen oder Markieren von Todos, etc.}
\item{Setzen von Hyperlinks (für Web-Projekte)}
\end{itemize}

Office-Dateien sind einfach auszutauschen – in Unternehmen werden die Dateien in der Regel auf einem Netzwerk-Laufwerk gespeichert. Zum Bearbeiten legt man sich eine lokale Kopie an und arbeitet in dieser Datei. Anschließend kopiert man die neue Version, meist unter Einhaltung eines bestimmten Benamungsschemas, wieder auf dem Netzlaufwerk ab. Hier können Konflikte auftreten (wenn zwei Personen gleichzeitig an der selben Datei arbeiten), diese müssen dann manuell gelöst werden. Wird die Datei direkt vom Netzlaufwerk geöffnet wird diese gesperrt und kann nur von einer Person bearbeitet werden.

Aufgrund der scheinbaren Vorteile der Office-Suite wird diese zu Beginn eines Projektes als geeignet angesehen und als Werkzeug für die Erfassung, Definition und Übersetzung der Texte eines Projektes ausgewählt.

Die im Verlauf des Projekts auftauchenden Probleme werden dann als gegeben akzeptiert, da man \typoquotes{nun} damit zurecht kommen muss, um den Verlauf des Projektes nicht zu verzögern. Bei neuen Projekten wird aber die gleiche Entscheidung wieder getroffen.

\subsection{Office-Programme sind die falschen Werkzeuge}

Der Grund, warum Office-Programme wie \emph{Word} und \emph{Excel} verwendet werden ist der, dass keine keine dedizierten Lösungen existieren, die explizit die genannten Abläufe in der Textverarbeitung abbildet. Es existieren vielen Produkte aus dem Bereich der Projektverwaltungswerkzeuge, Mediendatenbanken oder Content-Management-Systemen die die Prozesse rund um die Erstellung von \ac{IKTM} vereinfachen, aber keine kann die genannten Probleme und Abläufe zufriedenstellend abbilden.

\subsection{Beispiele aus der Praxis}

Die Analyse des Problems basiert auf Interviews mit Menschen, die in ihrem Arbeitsalltag regelmäßig mit Texten zu tun haben. In diesen Interviews wurden die Personen nach ihren Erfahrungen in der Projektarbeit bezüglich Texten befragt und gebeten die aus ihrer Sicht am häufigsten auftretenden Probleme zu nennen.

\subsubsection{MAN Truck \& Bus AG: Texte für mobile Vertriebssoftware}

Markus Rüb ist als Projektleiter bei der MAN Truck \& Bus AG mit der Einführung von Tablet-PCs als Vertriebshilfsmittel betraut.

\subsection{Schlussfolgerung}

