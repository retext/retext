\section{Anforderungen}\label{l:anforderungen}

Aus den Ausführungen in der Analyse des Problems in Abschnitt~\ref{l:problemanalyse}~(S.~\pageref{l:problemanalyse}~ff.), der Beschreibung der Personas in Abschnitt~\ref{l:personas}~(S.~\pageref{l:personas}~ff.) und der Analyse des Workflows der in Projekten typischerweise abläuft lassen sich funktionale Anforderungen an die Anwendung ableiten, die in diesem Abschnitt beschrieben werden. 

\subsection{Funktionale Anforderungen}

\begin{samepage}
In Abschnitt~\ref{l:workflow} wurde gezeigt, welche Operationen Mitarbeiter in Projekten ausführen, in denen Informations- und Kommunikationsmedien erstellt werden:

\begin{itemize}\itemsep -5pt
\item Definition
\item Befüllung
\item Korrektur
\item Kontrolle
\item Freigabe
\item Veröffentlichung
\end{itemize}
\end{samepage}

\begin{samepage}
und dass diese Operation Eigenschaften von Texten manipulieren, die sich in drei Bereiche unterteilen lassen:

\begin{itemize}\itemsep -5pt
\item Inhalt
\item Attribute, namentlich Identifier, Klasse, Textlänge und Position
\item Status
\end{itemize}
\end{samepage}

Die Anwendung muss also geeignete Funktionen anbieten, die mindestesn diese Operationen ermöglichen.

\TODO

\paragraph{Typ} Überschrift, Untertitel, Bild-Beschreibung, Fließtext.

\paragraph{Gleichzeitiges Bearbeiten von Texten} Es soll möglich sein, dass alle  Mitarbeiter gleichzeitig an den Texten eines Produktes arbeiten.


Aufteilen der Texte in einzelne Bausteine um diese eindeutig identifizieren zu können. Dies verhindert Copy\&Paste-Fehler (vgl. S.~\pageref{p:serielles-konzept}).

\label{l:hierarchien} Hierarchien sind aber in allen Produkten vorhanden und ein natürlicher Weg, Informationen zu gliedern. 

Fallback-Texte

\paragraph{Schnittstellen}

Anforderungen, Umfang, Ausprägung für Import-, Export- und Benachrichtigungsschnittstellen

Anbindung via CMIS http://en.wikipedia.org/wiki/Content\_Management\_Interoperability\_Services

Export eines Text-Booklets für die Rechtschreibkontrolle. Identifier mit ausgeben, um Texte dann schnell finden zu können. Hier könnte man auch einen QR-Code drucken, dann kann man mit einer mobilen App den Text direkt ändern.

Anbinden von Bilddatenbanken um projektspezifische Texte/Untertitel für Bilder zu definieren. Abgrenzung zu Video-Untertitel!

\subsection{Nicht-Funktionale Anforderungen}

\begin{itemize}
\item Zuverlässigkeit (Systemreife, Wiederherstellbarkeit, Fehlertoleranz)
\item Aussehen und Handhabung (Look and Feel)
\item Benutzbarkeit (Verständlichkeit, Erlernbarkeit, Bedienbarkeit)
\item Leistung und Effizienz (Antwortzeiten, Ressourcenbedarf, Wirtschaftlichkeit)
\item Betrieb und Umgebungsbedingungen
\item Wartbarkeit, Änderbarkeit (Analysierbarkeit, Stabilität, Prüfbarkeit, Erweiterbarkeit)
\item Portierbarkeit und Übertragbarkeit (Anpassbarkeit, Installierbarkeit, Konformität, Austauschbarkeit)
\item Sicherheitsanforderungen (Vertraulichkeit, Informationssicherheit, Datenintegrität, Verfügbarkeit)
\item Korrektheit (Ergebnisse fehlerfrei)
\item Flexibilität (Unterstützung von Standards)
\item Skalierbarkeit (Änderungen des Problemumfangs bewältigen)
\item Randbedingungen
\end{itemize}
