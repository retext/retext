\subsection{Anforderungen}\label{l:anforderungen}

Wie in der Schlussfolgerung in Abschnitt~\ref{l:schlussfolgerung} bereits erwähnt ergeben sich aus den genannten Problemen im vorangegangenen Kapitel die folgenden Anforderungen an eine Lösung.

\subsubsection{Funktionale Anforderungen}

\TODO

\paragraph{Gleichzeitiges Bearbeiten von Texten} Es soll möglich sein, dass alle  Mitarbeiter gleichzeitig an den Texten eines Produktes arbeiten.


Aufteilen der Texte in einzelne Bausteine um diese eindeutig identifizieren zu können. Dies verhindert Copy\&Paste-Fehler (vgl. S.~\pageref{p:serielles-konzept}).

\label{l:hierarchien} Hierarchien sind aber in allen Produkten vorhanden und ein natürlicher Weg, Informationen zu gliedern. 

Fallback-Texte

\paragraph{Schnittstellen}

Anforderungen, Umfang, Ausprägung für Import-, Export- und Benachrichtigungsschnittstellen

Anbindung via CMIS http://en.wikipedia.org/wiki/Content\_Management\_Interoperability\_Services

Export eines Text-Booklets für die Rechtschreibkontrolle. Identifier mit ausgeben, um Texte dann schnell finden zu können. Hier könnte man auch einen QR-Code drucken, dann kann man mit einer mobilen App den Text direkt ändern.

Anbinden von Bilddatenbanken um projektspezifische Texte/Untertitel für Bilder zu definieren. Abgrenzung zu Video-Untertitel!

\subsubsection{Nicht-Funktionale Anforderungen}


