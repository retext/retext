\section{Konzeption einer an den spezifischen Workflow angepassten Anwendung}\label{l:konzeption}

Im Kapitel \ref{l:problemanalyse} (S.\pageref{l:problemanalyse} ff.) wurde analysiert, welche Probleme in Zusammenhang mit der Produktion von Informations- und Kommunikationsmedien bezüglich den verwendeten Texten entstehen, wenn Textverarbeitungs- und Tabellenkalkulationsprogramme wie \trademark{Microsoft} \emph{Word} und \trademark{Excel} verwendet werden. Aus dieser Analyse wurde die Schlussfolgerung gezogen, dass eine browserbasierte Web-Anwendung am besten geeignet ist, den Workflow und alle Beteiligten zu integrieren. Dies wird in Abschnitt \ref{l:loesungsart} ausführlich begründet. Im vorangegangenen Kapitel \ref{l:personas} (S.\pageref{l:personas} ff.) wurden Personas vorgestellt, die repräsentativ für die Benutzer der Anwendung stehen. Deren Anforderungen an das System und die Analyse des spezifischen Workflows in Abschnitt \ref{l:workflow} bilden die Grundlage für den Entwurf der Anwendung in Kapitel \ref{l:entwurf}.

\subsection{Anforderungen}

\TODO

Um zu verhindern dass diese Probleme auftreten bedarf es einer Lösung, die es ermöglicht, dass alle Beteiligten gleichzeitig an den Texten für ein Produkt arbeiten können, und dass die Texte fehlerfrei in das Endprodukt übertragen werden können. Hierbei ist wichtig, dass je nach Aufgabe unterschiedliche Zusatzinformationen zu den Texten hinterlegt werden können und umfangreiche Feedback-Funktionen existieren. Der Zugang zu dieser Lösung muss allen Personen zu jeder Zeit und ohne Hindernisse möglich sein. Browserbasierte Web-Anwendungen sind hierfür am besten geeignet. In Abschnitt \ref{l:konzeption} (S.\pageref{l:konzeption}) wird eine browserbasierte Web-Anwendung konzipiert, die die genannten Probleme löst. 


\subsection{Art der Anwendung}\label{l:loesungsart}

Die im vorangegenen Abschnitt beschriebenen Anforderungen an die Anwendung, besonders im Hinblick auf den universellen, jederzeitigen Zugriff von überall sind die entscheidenden Anforderungen, die für die konzeption des Systems als \emph{browserbasierte Web-Anwendung mit vollständiger Schnittstellen-Abdeckung} sprechen. 

\begin{figure}[htb]
\begin{center}
\includegraphics[width=0.65\textwidth]{media/ArtdesSystems.pdf}
\caption{Aufbau des System in stark vereinfachter Darstellung}
\label{chart:aufbaudessystems}
\end{center}
\end{figure}

Diese Klasse von Anwendung verwendet einen Webbrowser als Laufzeitumgebung. Dabei stellt der Browser das GUI der Anwendung mit Hilfe von HTML, CSS und JavaScript dar, die Businesslogik und die Datenhaltung wird auf einem Server ausgeführt, mit der die GUI mithilfe einer Schnittstelle kommuniziert. Abbildung~\ref{chart:aufbaudessystems} / S.\pageref{chart:aufbaudessystems} zeigt den Aufbau des Systems in stark vereinfachter Darstellung. War es in den letzten Jahren noch üblich, dass Fragmente des GUIs mit serverseitigen Template-Sprachen erzeugt wurden (vgl.~\cite[S.48]{dunkel2008systemarchitekturen}) hat die zunehmende Verbreitung von mobilen Clients ein Umdenken zur Folge. Zum einen stellen Desktop-Clients, mobile Browser-Clients und native Apps zwar die gleichen Daten eines Systems dar, verwenden dafür aber nicht zwangsläufig die gleiche GUI-Technologie. Zum anderen werden Clients immer leistungsstärker, selbst Einsteiger-SmartPhones haben inzwischen CPUs mit mindestens dreistelligem Megahertz-Wert. Diese Entwicklung führt gerade bei Web-Anwendungen, auch Rich Internet Applications (RIAs) genannt, zu der Idee, Architekturen zu entwickeln, bei denen serverseitig keine GUI-Komponenten mehr erzeugt werden (vgl.~\cite{maccaw2011javascript} und \cite{coates2012phptemplating}). Clients kommunizieren über Schnittstellen mit dem Server und tauschen nur noch reine Daten aus. Dies hat mehrere Vorteile. Zum einen muss serverseitig kein Modell der clientseitigen Darstellung verwaltet werden, zum anderen verkleinert sich die Menge der transferierten Daten zwischen Client und Server erheblich. Dies hat besonders bei Benutzern mit langsamen oder schlechten Datenverbindungen im Mo"-bil"-funk-Netz große Vorteile. Für Web-Anwendungen bedeutet dass diese das zur Darstellung benötigte HTML mit Hilfe von JavaScript selber direkt im Client erzeugen. Beim ersten Besuch einer Internetseite müssen lediglich einmal die JavaScript-Dateien und benötigte statische Ressourcen wie CSS-Dateien, Bilder und ein statischer HTML-Grundaufbau geladen werden. Anschließend werden nur noch die für die jeweilige Aktion benötigten Daten mit Hilfe von JavaScript zwischen der Anwendung und dem Server ausgetauscht. Mobile Endgeräte, die über eigene GUI-Toolkits verfügen, oder Software von Drittanbietern können dann die selben Schnittstellen verwenden, ohne dass serverseitige Anpassungen vorgenommen werden müssen.

Web-Anwendung haben den Vorteil, dass sie ohne Installation auf dem Rechner des Benutzers lauffähig sind. Sie können als unmittelbar verwendet werden. Kompaitibilätsprobleme mit alten Browser-Versionen (z.B. dem \trademark{Internet Explorer 6}) können inzwischen mit Hilfe des \trademark{ChromeFrame}\footnote{\url{https://developers.google.com/chrome/chrome-frame/}} komfortable umgangen werden. Der Umfang an frei verfügbaren Bibliotheken zur Erstellung attraktiver und angenehm benutzbarer Anwendungen auf Basis von HTML ist riesig. Web-Anwendungen können mit wenig Aufwand auch auf mobilen Endgeräten eingesetzt werden, da Technologien zur platformabhängigen Anpassung der Darstellung (z.B. CSS-Mediaqueries) existierten. Insgesamt sind Webbrowser der aktuellen Generation mächtige Werkzeuge zur Erstellung von CRUD-Anwendungen. \cite{ms-key-software-development-trends} Die allgemeinen Vorteile einer browserbasierten Software, auch als \emph{Software as as Service} (SaaS) oder \emph{Application Service Provider} (ASP) Modell bekannt, liegen auf der Hand und werden an dieser Stelle nicht detailliert ausgeführt. Um nur einen zu nennen: die Möglichkeit, die Software jederzeit und für alle Mitarbeiter gleichzeitg ohne deren Zutun aktualisieren zu können, eliminiert vielfältige Probleme, die sonst in Umgebungen entstehen, in denen unterschiedliche Programmversionen existieren.

\subsection{Schnittstellen} 

Die Verwenden einer einheitlichen Schnittstelle durch alle Clients ermöglicht ein konsistentes Verhalten der Anwendungen über alle Zugangswege hinweg und ist besonders in Fall dieser Anwendung von Bedeutung, da die Benutzer des Systems wünschen, dass sich die Texte direkt innerhalb ihrer bevorzugten Werkzeuge abrufen und einbinden lassen. Dies ist nur mit Hilfe von Plugin-Ins möglich, die in der jeweiligen Umgebung der Software entwickelt werden müssen. Aus diesem Grund ist es ungvermeidlich, dass für alle Funktionen des Systems eine öffentliche Schnittstelle existiert.

Als Protokoll zur Kommunikation zwischen Clients und Server hat sich REST bewährt. Die Struktur des Protokolls ist direkt mit dem HTTP-Protokoll vebunden, so ist die Verarbeitung von REST-Anfragen serverseitig leicht mit Web-Frameworks zu implementieren, da diese von sich aus bereits für diese Art von Anfragen ausgelegt sind. Clientseitig wird lediglich ein HTTP-Client benötigt sowie Module zum Parsen von JSON- oder XML-Da"-ten"-struk"-tu"-ren -- Voraussetzungen, die von Browsern und SmartPhones erfüllt werden. JSON hat im Vergleich zu SOAP den Vorteil, dass es nicht versucht die Architektur der zugrundeliegenden Software nach außen abzubilden, so muss sich der Client nicht an bestimmte Reihenfolgen im Aufruf von Methoden halten. In der REST-Welt sind alle Operationen atomar und können ohne Vorbedingung gestellt werden. In der Praxis ist dies nicht immer umsetzbar, REST fordert serverseitig Zustandslosigkeit, die aber bei Systemen in denen Daten gespeichert und verändert werden nicht realisierbar ist. Aufgrund seines flexibleren Aufbaus, der Möglichkeit ausgewählte Anfragen leicht mit HTTP-Caches zu beschleunigen und der freien Wahl der Nachrichtenformats ist REST aus Sicht des Autors die besser Wahl zur Implementierung der Schnittstellenkommunikation.

\subsection{Zugang}

Diese Konzeption macht es möglich, jedem Mitarbeiter den passenden Zugang zu ermöglichen, im Einzelnen sind das:

\paragraph{Browserbasierter Zugang vom Desktop} Den Webbrowser werden alle Mitarbeiter verwenden, da in diesem GUI alle Funktionen der Anwendung implementiert sind.

\paragraph{Browserbasierter Zugang von SmartPhones} Auf gängigen SmartPhones sind Browser vorhanden, die in der Lage sind, die selben Inhalten anzuzeigen, wie ihr Desktop-Equivalent. Aufgrund der deutlich kleineren Bildschirmgröße und dem fehlen einer Maus ist es aber sinnvoll, dem Rechnung zu tragen und eine angepasste Version der Anwendung für diese Geräte bereit zustellen. 

\paragraph{Zugang direkt über die Schnittstellen} Vor allem im Bereich der Software-Entwicklung wird der direkte Zugriff der Entwickler auf die Schnittstellen der Anwendung eine wichtige Rolle spielen. So können diese die Integration des Systems in ihren Entwicklungsprozess optimal an die jeweiligen Bedürfnisse anpassen.

\paragraph{Exporte} Die Möglichlichkeit, die Texte des Projektes in verschiedene Formate zu exportieren ist eine wichtige Funktion. Sie ermöglicht zum einen die Übernahme in System und Programme, deren Anbindung nicht möglich oder gewünscht ist und schafft zum anderen die Möglichkeit, ähnlich wie Schnittstellen, die verarbeiteten Daten auf eine Art und Weise zu verwenden, die nicht vorhergesehen wurde oder nicht im Sinne der Anwendung liegt.

\paragraph{Zugang über Plug-Ins} Besonders für Mitarbeiter in der Produktion kann es wichtig sein, auf ihre angestammten Werkzeuge nicht verzichten zu müssen. Plug-Ins, also Erweiterungen für diese Werkzeuge, integrieren dann Teile der Funktionen der Anwendung in diese Werkzeuge. Beispielsweise könnte es ein Plug-In für \trademark{Adobe InDesign} ermöglichen, auf die Texte aus dem System zuzugreifen und diese direkt in Text-Rahmen im \trademark{InDesign}-Dokument zu übernehmen, so dass Copy\&Paste der Texte aus der Web-Anwendung oder einem exportierten Dokument entfallen kann.

\paragraph{Benachrichtigungen} Benachrichtigungen sind eine Form des Zugangs, die es Benutzern des Systems ermöglicht, über bestimmte Ereignisse informiert zu werden. Benachrichtigungen können in Form von E-Mails, SMS, über soziale Netzwerke wie Twitter, über Chat-Dienste wie Skype, IRC statt finden. Hierbei werden Information zu einem Ereignis übertragen z.B. der Status-Änderungen eines Textes. Denkbar sind auch auch Push-Exporte der Texte aufgrund eines bestimmten Ereignisses, z.B. der FTP-Export der Texte als CSV-Datei, sobald das Projekt abgschlossen ist oder Änderungen freigegeben wurden.

\paragraph{Software von Drittanbietern} Dadurch, dass alle Funktionen der Anwendung über eine Api exponiert werden, sind auch fremde Softwarehersteller in der Lage, Teile der Funktionen oder die gesamte Anwendung in einer eigenen GUI zu implementieren. So können Bedürfnisse von Anwendergruppen mit besonderen Anforderungen abgedeckt werden, die bei der Konzipierung der Anwendung nicht berücksichtigt wurden.

\subsection{Überblick über den Aufbau des gesamten Systems}

Abbildung~\ref{chart:gesamtessystem} liefert einen Überblick über den Aufbau des gesamten Systems:

\begin{figure}[htb]
\begin{center}
\includegraphics[width=\textwidth]{media/GesamtesSystem.pdf}
\caption{Aufbau des gesamten Systems im Überblick}
\label{chart:gesamtessystem}
\end{center}
\end{figure}

Die Zentrale Komponente der Anwendung bildet der Server. Für die Benutzer erfolgt der Zugriff mit Hilfe einer GUI, die mit der REST-API des Servers kommuniziert. Eine browserbasierte GUI auf Basis von HTML5 und JavaScript bildet den Hauptzugang zum System, der auch auf SmartPhones verwendet werden kann. Zusätzlich gibt es spezielle Plugins für Adobe-Produkte und weitere wichtige Produktionsumgebungen. Auch native GUIs für Smartphones verwenden die gleiche API. Die Schnittstellen können auch von Drittanbietern dazu verwendet werden, eigenen Clients für das System zu entwickeln. In die Endprodukte gelangen die Texten über den Export, exportiert wird dabei in viele Formate, neben Datenformaten wie z.B. XML werden auch Dokumentenformate wie z.B. Word exportiert. Der Export kann durch den Anwender erzeugt werden (\emph{Pull-Export}), aber auch automatisch, z.B. nach festgelegten Zeitplänen oder Ereignissen erfolgen. Dieser \emph{Push-Export} erfolgt auf je nach Projekt festlegbaren Orte, wie z.B. FTP-Server oder Versionsverwaltungssysteme. Die Benachrichtigungen über Aufgaben und Änderungen an Texten kann via E-Mail, aber auch mittels Instant-Messaging-Systeme oder durch den Aufruf fremde API-Endpunkte erfolgen.

\subsection{Der spezifische Workflow}\label{l:workflow}

Beobachtet man verschiedene Projekte, in denen Informations- und Kommunikationsmedien erstellt werden, lässt sich feststellen, dass Texte immer wieder auf die gleiche Art beeinflusst werden. Für eine vollständige Beschreibung des Workflows ist es zunächst sinnvoll, zu ermitteln, \emph{wie} Texte beeinflusst werden. 

% MARK

\begin{figure}[htb]
\begin{center}
\includegraphics[width=\textwidth]{media/chart-3.pdf}
\end{center}
\caption{Operationen bei der Erstellung von Texten}
\label{chart:3}
\end{figure}

Betrachtet man die Arbeiten in Zusammenhang mit Text lassen sich diese in sechs eigenständige Operationen unterteilen:

\begin{enumerate}
\item{Durch \textbf{Definieren eines Textbausteines} wird festgelegt, wie der benötigte Text beschaffen sein muss. Die Aussage \typoquotes{Wir brauchen an dieser Stelle eine Überschrift} ist ein Beispiel für diese Operation. Sie legt fest, wie der Textbausteine gestaltet werden muss, um die ihm zugedachte Aufgabe zu erfüllen. Neben der Angabe zur Platzierung auf dem Medium durch \typoquotes{an dieser Stelle} wird implizit durch \typoquotes{eine Überschrift} eine Angabe zur inhaltlichen und visuellen Gestaltung getroffen; Überschriften sollen kurz und knapp sein und ihre visuelle Gestaltung wird durch den Styleguide des Projektes festgelegt.}
\item{Das \textbf{Schreiben eines Textes} befüllt einen Textbaustein mit einem Text in einer Sprache. Bei diesem Vorgang wird der Text entsprechend der Vorgabe aus der Beschreibung als Original erstellt oder aus Quellen außerhalb des Projektes kopiert und eingefügt. }
\item{In der \textbf{Korrektur} wird der Text inhaltlich und grammatikalisch überprüft und entsprechend angepasst. Der Korrektor muss dabei für eine grammatikalische Überprüfung des Textes kein Fachwissen bezogen auf das Projekt haben. Ist diese Fachwissen vorhanden, kann eine inhaltliche Korrektur vorgenommen werden.}
\item{In der \textbf{Qualitätskontrolle} wird der Text dahingehend überprüft, ob er den Anforderungen gemäß der Beschreibung und inhaltlichen Vorgaben, auch hinsichtlich des gesamten Projektes entspricht. }
\item{Durch die \textbf{Freigabe} wird der Text abgenommen und kann nun in das Endprodukt übernommen werden.}
\item{Durch die \textbf{Veröffentlichung} wird der Text in das Endprodukt eingebracht.}
\end{enumerate}

\begin{figure}[htb]
\begin{center}
\includegraphics[width=\textwidth]{media/chart-4.pdf}
\end{center}
\caption{Operationen bei der Erstellung von Texten mit Qualitätskontrolle}
\label{chart:4}
\end{figure}

Diese Operationen werden auch 1:1 auf die übersetzte Version eines Textes angewendet.

\begin{figure}[htb]
\begin{center}
\includegraphics[width=\textwidth]{media/chart-5.pdf}
\end{center}
\caption{Operationen bei der Übersetzung von Texten mit Qualitätskontrolle}
\label{chart:5}
\end{figure}

In Abschnitt~\ref{l:besondererolle} ab Seite~\pageref{l:besondererolle} wurde beschrieben, wie umfangreich die Anzahl der Personen ist, die Einfluss auf die Texte eines Produktes haben. Die Rollenverteilung ist dabei von Projekt zu Projekt unterschiedlich. Allen gemeinsam ist aber, dass die beteiligten Personen  Einfluss auf drei grundlegenden Eigenschaften von Text haben: den Inhalt des Textes, die Attribute wie z.B. \typoquotes{maximale Textlänge} oder \typoquotes{Position im Medium} und den Status wie z.B. \typoquotes{neu} und \typoquotes{freigegeben}. Anhand dieses Kriteriums lassen sich Mitarbeiter in drei Gruppen unterteilen:

% MARK

\paragraph{Personen, die Einfluss auf den Inhalt haben}

\paragraph{Personen, die Einfluss auf den Attribute haben}

\paragraph{Personen, die Einfluss auf den Status haben}




\begin{enumerate}
\item{Der \textbf{Informationsarchitekt} (oder Konzepter) legt die Struktur eines Produktes fest und damit auch die Art und Menge des benötigten Textes,}
\item{der \textbf{Texter} verfasst die Texte,}
\item{der \textbf{Übersetzer} überträgt die Texte in weitere Sprachen,}
\item{der \textbf{Qualitätsmanager} überwacht die Ergebnisse der Prozesse,}
\item{der \textbf{Produktbesitzer} (oder Kunde) ist für die fachlichen und rechtliche Aspekte, sowie das Festlegen der zeitlichen Rahmenbedingungen verantwortlich,}
\item{der \textbf{Produzent} ist für die Erstellung des eigentlichen Produktes verantwortlich.}
\end{enumerate}

Alle Rollen haben im Verlauf eines Projekts, zu unterschiedlichen Zeiten und mit unterschiedlichem Gewicht, Einfluss auf die Gestaltung der Texte. Es existieren auch Abhängigkeiten zwischen den Rollen, so kann ein Übersetzer erst arbeiten, wenn der Text vorliegt und vom Produktbesitzer abgenommen wurde; wird aber zu einem späteren Zeitpunkt der Text geändert, muss auch wieder der Übersetzer neu beginnen.


\subsection{Zusammenfassung, Nachteile \& Risiken des Konzepts}

In diesem Abschnitt wurde eine Anwendung konzipiert und der darin abgebildetet Workflow beschrieben. Die Konzipierung der Anwendung als Web-Anwendung, bei der alle durchführbaren Operationen über Schnittstellen abgedeckt sind, ermöglicht es, für jeden Mitarbeiter die passenden Zugangswege anzubieten. Als Hauptzugang wird der Webbrowser verwendet, so ist sichergestellt, dass alle Mitarbeiter alle Funktionen des Systems ohne zusätzliche Aufwände wie die Installation neuer Software verwenden können. Für spezielle Anwendungsfällen ist es mit Hilfe der API möglich, Plug-Ins zu entwickeln, die sich in die bevorzugten Werkzeuge der Anwender integrieren.

\paragraph{Nachteile \& Risiken} Ein Nachteil dieses Konzepts liegt in der Zentralisierung der Datenspeicherung. Da alle Daten auf einem zentralen Server verwaltet werden, ist dieser auch der \emph{Single Point of Failure}, d.h. sollte der Server ausfallen, kann kein Mitarbeiter weiterarbeiten. Für einen kommerziellen Betrieb eines solchen Systems ist es also unabdingbar, dass die Server-Infrastruktur ausfallsicher konzipiert ist. 

Das Übertragen der Daten auf einen zentralen Server kann auch zu Bedenken bei den beteiligten Unternehmen führen. Es gibt gerade bei größeren Unternehmen Vorbehalte dagegen, Informationen auf Systemen von Drittanbietern zu speichern. Hier gilt es, genau wie im Hinblick auf die Verfügbarkeit des Systems, einen vertrauenswürdigen Betreiber für die Server-Infrastruktur zu finden. Alternativ ist es jedoch problemlos möglich, das System auch \emph{In-House}, also auf Servern im Unternehmen als \emph{Appliance}, zu betreiben, wobei dann aber zusätzliche Wartungsaufwände entstehen, und damit einige Vorteile des SaaS-Modells ausgehebelt werden. 

Da alle Mitarbeiter über das Internet mit der Anwendung verbunden sind, spielt auch die Bandbreite und Verfügbarkeit einer Internetverbindung eine Rolle. Im Unternehmensbereich spielt dies aber inzwischen nur nur eine untergeordnete Rolle. Trotzdem sollten geeignete Maßnahmen ergriffen werden, die die Arbeit auch mit einer langsamen oder sogar ganz ohne eine Internetverbindung ermöglicht (Offline-Access).

Das größte Risiko dieses Konzeptes ist, dass Mitarbeiter gezwungen werden, sich von ihren bekannten Werkzeugen zu lösen. Gerade bei Mitarbeitern, die vor allem mit Textverarbeitungsprogrammen arbeiten und ansonsten kaum mit anderen Werkzeugen Kontakt haben, wird der Umstieg von der unstrukturierten Arbeitsweise in \trademark{Word} auf die, bis auf den einzelnen Text heruntergebrochene Arbeitsweise in der vorgschlagenen Anwendung, schwer fallen. Man kann aber davon ausgehen, dass für alle Beteiligten die Vorteile der Lösung erkenntlich werden und sich eine Abneigung gegen eine Änderung angestammter Arbeitsabläufe leicht abbauen lässt.

\bigskip

Im nächsten Kapitel wird eine Anwendung entworfen, die dieses Konzept umsetzt.

\pagebreak